\documentclass[master,german]{hgbthesis}
% Zulässige Class Options: 
%   Typ der Arbeit: diplom, master (default), bachelor, praktikum 
%   Hauptsprache: german (default), english
%%------------------------------------------------------------

\graphicspath{{images/}}    % name of directory containing the images
\logofile{logo.png}			% name of PDF, remove or use \logofile{} for no logo
\bibliography{literatur}  	% name of the BibTeX (.bib) file

%%%----------------------------------------------------------
%% Custom Macros
%%%----------------------------------------------------------

\newcommand{\enword}[1] {\textit{#1}}
\newcommand{\todo}{\textcolor{red}{\textbf{TODO}}}

 % Keine "Schusterjungen"
 \clubpenalty = 10000
 % Keine "Hurenkinder"
 \widowpenalty = 10000 \displaywidowpenalty = 10000

\sloppy 

%%%----------------------------------------------------------
\begin{document}
%%%----------------------------------------------------------

% Einträge für ALLE Arbeiten: --------------------------------
\title{\textbf{Rayden} \\--- \\Ein System für funktionale Tests mit Spezialisierung auf Abnahmetests}
\author{Thomas Fischl}
\studiengang{Software Engineering}
\studienort{Hagenberg}
\abgabedatum{2015}{06}{17}	% {YYYY}{MM}{DD}

%%% zusätzlich für eine Bachelorarbeit: ---------------------
\nummer{1310454009}   % XX...X = Stud-ID, z.B. 0310238045-A  
                        % (A = 1. Bachelorarbeit)
\semester{Sommersemester 2015} 
\gegenstand{Formale Sprachen, Compiler- und Werkzeugbau} 
\betreuer{FH-Prof. DI Dr. Heinz Dobler}

%%\strictlicense  % erzeugt restriktive Lizenzformel

%%%----------------------------------------------------------
\frontmatter
\maketitle
\tableofcontents
%%%----------------------------------------------------------

\chapter{Dank}

Zu Beginn möchte ich mich ganz herzlich bei meinem Betreuer Herrn FH-Prof. DI Dr. Heinz Dobler bedanken. Herr Dobler hat mich bereits bei der Bachelorarbeit betreut und ich bin froh, dass er mich auch wieder bei der Masterarbeit betreut hat. Seine Kompetenz im Bereich des Compilerbaus und sein Verständnis für berufsbegleitend Studierende hat zum Erfolg dieser Masterarbeit erheblich beigetragen. 

\SuperPar
Weiters möchte ich mich bei meinen Eltern und bei meiner Schwester bedanken. Sie haben mich immer ermutigt weiter zu machen und haben mich in allen meinen Entscheidungen unterstützt. Bei meiner Schwester möchte ich mich auch noch speziell für das Design des Rayden-Logos bedanken.

\SuperPar
Aber am meisten möchte ich mich bei meiner Freundin Angelika bedanken. Sie war die gesamte Studienzeit für mich da und hat mich tatkräftig unterstützt. Ich bedanke mich vor allem für das große Verständnis, da ich an vielen Wochenenden mehr Zeit mit dem Studium als mit ihr verbracht habe. Danke.

\chapter{Kurzfassung}

In den letzten Jahren ist das Automatisieren von Tests wieder in den Fokus von Testmanagerinnen und Testmanagern gerückt. Gründe dafür sind die Verkürzung der Release-Zyklen und ein immer größerer Kostendruck. Daher stehen viele Testabteilung vor dem Problem ihre manuellen Tests zu automatisieren.

\SuperPar
Ein Lösungsansatz dafür ist der \enword{Keyword-Driven-Testing}-Ansatz, welcher sich in letzter Zeit großer Beliebtheit erfreut. Für diesen Testansatz wurden einige Open-Source-Anwendungen aber auch kommerzielle Lösungen entwickelt. Jedoch hat dieser Ansatz neben vielen Vorteile auch einige Nachteile. Je größer die Projekt werden, desto schwieriger wird die Verwaltung da es nur wenig Werkzeuge für diese Anwendungen gibt. 

\SuperPar
Dieser Ausgangspunkt stellt die Motivation für diese Masterarbeit. In dieser Mastarbeit wird ein neues System entwickelt, welches den \enword{Keyword-Driven-Testing}-Ansatz umsetzt. Jedoch setzt diese Lösung auf einen Compiler um eine bessere Unterstützung für die Verwenderinnen und Verwender liefern zu können. In das System wird auch das Konzept eines \enword{Object Repositories} integriert. Das Konzept soll dabei helfen, Abnahmetests leichter und besser lesbar zu schreiben.

\SuperPar
Um die Fähigkeiten dieses neu entwickelten Systems zu zeigen, wird einen Webanwendung mit diesem System getestet. Dabei wird gezeigt, wie man unterschiedliche Testmethoden mit diesem System vereinen kann und welche besonderen Stärken im Bezug auf Abnahmetests existieren.		
\chapter{Abstract}

\begin{english} %switch to English language rules

TODO !!!

%und hier geht dann das Abstract weiter...
\end{english}
			

%%%----------------------------------------------------------
\mainmatter         % Hauptteil (ab hier arab. Seitenzahlen)
%%%----------------------------------------------------------

\chapter{Einleitung}
\label{cha:Einleitung}

\section{Zur Projektbezeichnung Rayden}

Der Begriff \enword{Rayden} ist abgeleitet von dem japanischen Wort \enword{Raijin}, welches im japanischen Volksglauben der Name des Donner-Gotts ist. In der westlichen Welt wird der Name aber meist \enword{Raiden} geschrieben, woraus für diese Arbeit die Bezeichnung \enword{Rayden} abgeleitet wurde. Die Abbildung \ref{fig:logo} zeigt das Logo für Rayden, welches von Agnes Fischl gestaltet wurde.

\begin{figure}[h]
\centering
\includegraphics[width=0.3\textwidth]{logo.png}
\caption{Rayden-Logo}
\label{fig:logo}
\end{figure}


%%------------------------------------------------------------------------------------------------------

\section{Motivation}

Viele Softwareunternehmen haben in den letzten Jahren große Testabteilungen aufgebaut. Der Fokus in diesen Abteilungen liegt sehr häufig noch auf dem manuellen Testen der Benutzeroberflächen. Dabei müssen für jede neue Version der Software viele manuelle Schritte durchlaufen werden. Dieser Vorgang ist sehr zeit- und kostenintensiv. Durch den Vormarsch neuer Entwicklungsmethoden und wegen des starken Kostendrucks stehen viele dieser Abteilungen vor einem Problem. Auf der einen Seite müssen sie Kosten einsparen, auf der anderen Seite werden die Release-Zyklen immer kürzer, was einen noch größeren Aufwand bedeutet. In diesem Spannungsfeld überlegen viele dieser Unternehmen, ihre manuellen Tests zu automatisieren, um dadurch langfristig Zeit und Geld zu sparen. 

\SuperPar
Dieser Transformationsprozess stellt die Organisationen vor große Herausforderungen. Sie haben tausende Stunden von Expertenwissen in die manuellen Tests investiert. Für die Automatisierung steht jedoch selten derselbe Umfang an Zeit und Geld zur Verfügung. Auch muss der Prozess meistens parallel zu den bestehenden manuellen Tests vollzogen werden, da man kaum eine vollständige Umstellung auf einmal erledigen kann.

\SuperPar
Um diesen Prozess für die Testabteilung zu erleichtern, benötigt es ein mehrschichtiges Test-\enword{Framework}, das in der Lage sein muss, bestehende manuelle Tests wiederverwenden zu können. Dabei darf die Lesbarkeit der manuellen Tests aber nicht verloren gehen, da diese in Ausnahmefällen noch von einer Testerin oder einem Tester manuell durchgeführt werden müssen.

%%------------------------------------------------------------------------------------------------------

\section{Problemstellung}

Viele Testabteilungen arbeiten heutzutage noch größtenteils mit manuellen Tests. Diese Tests sind über Jahrzehnte gewachsen und es wurden tausende von Stunden in die Erstellung und Wartung investiert. Diese Testabteilungen bestehen aus vielen Testerinnen und Testern, welche die Tests für jede neue Version einer Software manuell ausführen. Es kommt nicht selten vor, dass aus Zeitgründen nicht alle Tests für jede Version ausgeführt werden können. Diese Situation hat sich durch den Einsatz von agilen Entwicklungsprozessen und kürzeren Release-Zyklen noch deutlich verschärft. 

\SuperPar
Diese Entwicklungen machen es notwendig, dass sich Testabteilungen immer öfter mit dem Thema der Testautomatisierung auseinandersetzen müssen. 

\SuperPar
Das führt zu folgenden Herausforderungen für die Testabteilungen:\\

\begin{enumerate}

\item Für die Automatisierung der Tests steht oft nur ein geringes Budget zur Verfügung.\\
	
\item Das Wissen aus den manuellen Test darf nicht verloren gehen.\\
	
\item Während des Migrationsprozesses und auch danach muss es möglich sein, dass man automatisierte Tests manuell ausführen kann. Diese Eigenschaft kann notwendig werden, um fehlgeschlagene automatisierte Testausführungen nachträglich manuell verifizieren zu können.\\

\item Die bestehenden Automatisierungslösungen sind oft sehr technisch aufgebaut. Jedoch findet man in typischen Testabteilungen nur wenige Entwicklerinnen und Entwickler, welche mit diesen Lösungen arbeiten können.

\end{enumerate}

\SuperPar
Um alle Herausforderungen dieser Liste zu adressieren, reicht \enword{eine} technische Lösung heutzutage nicht mehr aus. Ein Test-\enword{Framework} in diesem Umfeld muss auf vielen unterschiedlichen Ebenen ansetzen und Unterstützung bieten. 

\SuperPar
Eine mögliche Lösung für Testabteilungen wäre eine Testmethode, welche es auch Testerinnen und Testern ohne fundierte Programmierkenntnisse ermöglicht, automatisierte Tests zu erstellen. Zusätzlich muss es mit dieser Testmethode möglich sein, bestehende manuelle Tests wieder zu verwenden. Schlussendlich müssen die automatisierten Tests noch immer in einem Format vorliegen, das es ermöglicht, diese auch manuell von einer Testerin oder einem Tester auszuführen.

%%------------------------------------------------------------------------------------------------------

\section{Zielsetzung}

Das Ziel dieser Arbeit ist es, die Fähigkeiten eines \enword{Keyword-Driven-Testing}-Ansatzes mit einem \enword{Object Repository}  zu kombinieren. Im Zuge der Implementierung soll ein neues Test-\enword{Framework} entwickelt werden, welches den Ansatz von \enword{Keyword-Driven Testing} verwendet. Für das \enword{Framework} soll eine neue Sprache entwickelt werden, welche die Bedürfnisse nach einer einfachen und gut lesbaren Sprache befriedigt. Zusätzlich soll die Sprache eine gute Unterstützung für das \enword{Object Repository} liefern.

\SuperPar
Im nächsten Kapitel werden die Grundlagen und verwendeten Technologien detailliert beschrieben.
\chapter{Grundlagen und Technologien}
\label{cha:StandDerTechnik}

In diesem Kapitel werden die grundlegenden funktionalen Testmethoden beschrieben, welche in der Softwareentwicklung angewendet werden. Andere Testbereiche der Softwareentwicklung wie Performanztest und Penetrationstest werden hier nicht behandelt, denn diese vorliegende Arbeit zielt speziell auf die funktionalen Testmethoden ab und hat in diesem Bereich ihre Stärken, was aber nicht bedeutet, dass man dieselben Konzepte nicht auch für die anderen Testbereiche anwenden könnte. 


\begin{figure}
\centering
\includegraphics[width=0.9\textwidth]{testuebersicht.png}
\caption{Testmethoden unterteilt in \enword{White-Box-} und \enword{Black-Box}-Testen}
\label{fig:testtypen}
\end{figure}

\SuperPar
Funktionale Tests haben die Aufgabe sicherzustellen, dass die Anforderungen aus der Spezifikation korrekt umgesetzt werden. Für die Umsetzung dieser Tests können sowohl manuelle als auch automatisierte Testmethoden verwendet werden, welche man in Abbildung \ref{fig:testtypen} sieht. Die unterschiedlichen Methoden werden in diesem Kapitel genau beschrieben und es wird auf die Unterschiede eingegangen.

\SuperPar
Im zweiten Teil dieses Kapitels werden die verwendeten Technologien beschrieben. Dazu gehören die Werkzeuge und Bibliotheken, welche für die Entwicklung der Sprache Rayden und der Ausführungseinheit verwendet werden. Weiters werden zwei Testtreiber-Bibliotheken beschrieben, welche für automatisierte Abnahmetests verwendet werden können. 

\section{\enword{White-Box}-Test}

Unter \enword{White-Box}-Tests versteht man Tests, bei denen die Testerinnen und Tester Zugriff zu dem Quelltext haben. \enword{White-Box}-Tests werden speziell für bestimmte Codefragmente geschrieben und testen gezielt einzelne Fragmente einer Anwendung. Diese Tests werden in einer frühen Phase des Entwicklungsprozesses erstellt und liefern sehr bald Qualitätskennzahlen. 

\SuperPar
Diese Tests sind typischerweise sehr technisch und verlangen vom Testpersonal Programmierkenntnisse. Daher werden diese Tests von den Personen aus der Entwicklungsabteilung selbst geschrieben und fallen nicht in den Zuständigkeitsbereich des Qualitätssicherungsteams. Das gilt natürlich nur solange, als man sich nicht in einem agilen Entwicklungsprozess befindet. Dort werden sowohl die \enword{White-Box}- als auch die \enword{Black-Box}-Tests im Entwicklungsteam umgesetzt.


\section{\enword{Black-Box}-Test}

Die Gruppe der \enword{Black-Box}-Tests sind klassische Aufgaben eines Qualitätssicherungsteams. Diese Gruppe umfasst alle Testansätze, bei denen der Quelltext der Anwendung nicht vorliegt. Dabei ist die Anwendung eine \enword{Black Box}. Die Aufgabe des Qualitätssicherungsteams ist es, zu überprüfen, ob alle Anforderungen laut Spezifikation umgesetzt wurden. Für diese Aufgabe steht dem Qualitätssicherungsteam eine ganze Reihe an unterschiedlichen Ansätzen zur Auswahl, angefangen von manuellen Tests über explorative Tests bis hin zu automatisierten Abnahmetests. 

\SuperPar
Die \enword{Black-Box}-Tests werden typischerweise im fortgeschritten Projektstadium durchgeführt. Das ergibt sich aus der Tatsache, dass man für die \enword{Black-Box}-Tests eine lauffähige Anwendung benötigt.

\section{Manuelle Testmethoden}

Bei manuellen Tests handelt es sich generell um \enword{Black-Box}-Tests. Dabei überprüft die Testerin oder der Tester, ob sich die Anwendung in Bezug auf die in der Spezifikation angegebenen Anforderungen korrekt verhält und ob die Funktionalität vollständig vorhanden ist. Die Funktionalität wird typischerweise über die Benutzeroberfläche geprüft. Eine zusätzliche Aufgabe bei manuellen Tests ist es, zu überprüfen, ob die Benutzeroberflächen-Konzepte korrekt und einheitlich umgesetzt wurden.

\SuperPar
Bei manuellen Tests ist es typisch, dass die Testmanagerin oder der Testmanager eine textuelle Beschreibung der Testfälle erstellt. Diese Testfälle werden dann von dem Qualitätssicherungsteam für jede neue Version der Anwendung durchgeführt. Bei einer Abweichung der Anwendung muss entschieden werden, ob sich der Anwendungsfall geändert hat oder ob die Anwendung nicht korrekt funktioniert. Im letzteren Fall muss ein Fehlerbericht verfasst und an die Entwicklungsabteilung gesendet werden.

\subsection{Explorativer Test}

Eine Spezialform des manuellen Testens ist das explorative Testen. Dabei bekommt die Testerin oder der Tester keine genaue Vorgabe, wie ein Anwendungsfall getestet werden soll. In diesem Fall bekommt die Person nur eine Aufgabe gestellt, welche mit der Anwendung gelöst werden muss. Das Ziel ist es, dass man unterschiedlichste Möglichkeiten der Anwendung testen kann. Dieser Ansatz ist gut dafür geeignet, um neue Fehler zu finden. 

\SuperPar
Grundsätzlich haben manuelle und automatisierte Tests die Limitierung, dass nur festgestellt werden kann, ob eine neue Version einer Anwendung gleich gut funktioniert wie die alte. Es können aber keine neuen Fehler abseits der definierten Tests gefunden werden. Diese Lücke versucht das explorative Testen zu schließen. Es ist auch von Vorteil, wenn nicht immer die gleiche Person dieselbe Aufgabe testet. Jede Person hat neue Ideen, wie man die Aufgabe lösen kann und testet daher neue Bereiche und Kombinationen der Anwendung. Dieser Ansatz ist ein kreativer Prozess und kann daher im Gegenteil zu manuellen Tests nicht automatisiert werden.

\section{Automatisierte Testmethoden}

Das Ziel von automatisierten Tests ist es, dass man den Testaufwand in einem Software-Projekt reduziert. Aus wirtschaftlicher Sicht ist es viel besser, wenn das stupide Testen durch einen automatisierten Test erledigt wird. Dadurch reduzieren sich die Kosten für das Software-Projekt. Bei einer manuellen Ausführung kann es bei mehrmaligen Wiederholungen eines Tests zu Aufmerksamkeitsverlusten kommen, was bei automatisierten Tests nicht der Fall ist.

\SuperPar
Durch automatisierte Tests werden Qualitätssicherungsteams jedoch nicht obsolet. Auf der einen Seite müssen die automatisierten Tests auch von jemandem geschrieben und gewartet werden, auf der anderen Seite sind automatisierte Tests für exploratives Testen ungeeignet. Die Aufgabe des explorativen Testens wird auf absehbare Zeit immer durch eine Person erledigt werden.

\SuperPar
Schlussendlich gibt es noch einen weiteren wichtigen Vorteil von automatisierten Tests gegenüber manuellen Tests: Man kann automatisierte Tests viel öfter ausführen und sie liefern schneller eine Aussage über die Qualität der Software. Diese Zeitreduktion ist für agile Softwareprozesse sehr wichtig, denn damit bekommt das Entwicklungsteam schneller eine Rückmeldung darüber, ob das System noch korrekt funktioniert. 


\subsection{Komponententest (\enword{Unit Testing})}

Bei einem Komponententest \cite{xUnit} wird genau eine Softwarekomponente getestet. Eine Softwarekomponente ist eine abgeschlossene Einheit in einem Software-Projekt, welche eine definierte Schnittstelle hat. Das kann zum Beispiel eine einzelne Klasse, aber auch ein ganzes Modul sein, wie zum Beispiel in Pascal. Aus diesem Grund wird der Komponententest auch oft als Modul-Test oder Unit-Test bezeichnet. In dem Fall, dass die zu testende Komponente eine Abhängigkeit von einer anderen Komponente hat, werden diese durch eine Test-Implementierung ersetzt. Der Vorteil von Komponententests ist, dass deren Erstellung und Wartung keinen großen Aufwand verursachen. Das ist auch der Grund, warum dieser Testansatz sehr beliebt und weit verbreitet ist. Die Beliebtheit dieser Variante kann man daran ablesen, dass es für so gut wie jede Programmiersprache eine Unit-Test-Bibliothek wie zum Beispiel JUnit \cite{JUnit} gibt. Der Vorteil ist aber auch der größte Nachteil bei diesem Ansatz: Die Komponenten werden einzeln getestet und man kann daher keine Aussage darüber treffen, wie sich das Gesamtsystem verhalten wird.

\SuperPar
Um eine Aussage über das Verhalten des Gesamtsystems zu bekommen, kann man Integrationstests verwenden. Diese werden im nächsten Abschnitt erklärt.

\subsection{Integrationstest (\enword{Integration Testing})}

Der Integrationstest ist schon deutlich aufwendiger und umfangreicher als ein Komponententest. Bei einem Integrationstest werden alle Komponenten eines Softwaresystems gemeinsam getestet. Das Ziel bei diesen Tests ist es zu gewährleisten, dass alle Komponenten miteinander funktionieren und dass alle Schnittstellen korrekt implementiert wurden. Es werden auch unterschiedliche Fehlersituationen im System simuliert und überprüft, ob diese ausgeglichen werden können. Ein einzelner Fehler in einer Komponente soll nicht das ganze System zum Absturz bringen oder in einen ungültigen Zustand versetzen.

\SuperPar
Bei einem Integrationstest stellt sich oft die Frage, ob man mit oder ohne Datenbank testen soll. Diese Frage kann man nicht so einfach beantworten. Auf der einen Seite kann man sagen, dass die Datenbank genauso eine Komponente im Softwaresystem ist, welche getestet werden muss. Auf der anderen Seite kann man argumentieren, dass die Datenbank ein externes System ist, welches bereits getestet wurde. Grundsätzlich ist jedoch zu sagen, dass es ein guter Ansatz ist, wenn man mit einer Datenbank die Integrationstests durchführt. Eine ausführliche Diskussion über diese Thematik kann man im Buch \enword{Der Integrationstest} \cite{intTest} nachlesen. Es kann immer wieder vorkommen, dass genau bei der Schnittstelle zwischen Softwaresystem und Datenbank Probleme auftreten. Diese Fehler würden sonst erst relativ spät im Projekt-Lebenszyklus auftreten und der Aufwand für die Behebung dieser Fehler würde steigen.

\SuperPar
Der Grund, warum über dieses Thema so viel diskutiert wird ist, dass der Aufwand für einen Integrationstest mit Datenbank deutlich höher ist. Man muss eine Strategie überlegen, wie man für jede Testausführung einen definierten Datenbankzustand herstellen kann. Dieser Datenbankzustand ist sehr wichtig, um reproduzierbare Tests schreiben zu können. 


\subsection{Schnittstellentest (\enword{API Testing})}

Der Schnittstellentest ist die Vorstufe zum Abnahmetest. Dabei werden alle externen Schnittstellen getestet. Dabei kann es sich um eine Schnittstelle in ein externes System handeln oder um eine Web-Service-Schnittstelle. Aber darunter fällt auch die Schnittstelle zwischen Benutzeroberfläche und Geschäftslogik. Diese Schnittstelle ist für die Testautomatisierung sehr interessant, da man hierbei die Benutzeroberfläche nicht für das Testen benötigt, jedoch das Gesamtsystem testen kann. Der Vorteil liegt darin, dass dieser Testansatz deutlich stabiler ist als ein Abnahmetest, welcher die Tests über die Benutzeroberfläche ausführt. Auch ist die Durchlaufzeit eines Schnittstellentests deutlich geringer als bei einem Abnahmetest.

\SuperPar
Der Unterschied zwischen einem Schnittstellentest und einem Integrationstest ist, dass bei einem Schnittstellentest das Software-System vollständig installiert wird. Für die Tests wird eine vollwertige Datenbank mit realistischen Testdaten verwendet. Bei einem Integrationstest verzichtet man auf diesen Aufwand.

\SuperPar
Wie schon die vorhergehenden Testansätze hat auch dieser Ansatz einen großen Nachteil: Bei diesen Tests werden nur die Schnittstellen zwischen externem System und der Benutzeroberfläche getestet. Dabei kann aber nicht sichergestellt werden, dass die Benutzeroberfläche fehlerfrei funktioniert. Für die Benutzerin oder den Benutzer der Anwendung zählt schlussendlich nur, ob die Benutzeroberfläche korrekt funktioniert. Aus diesem Grund sind all diese Testansätze kein Ersatz für die Abnahmetests.

\subsection{Abnahmetest (\enword{User Acceptance Testing})}

Abnahmetests sind die aufwendigsten und kostenintensivsten Aufgaben im Testprozess. Bei einem Abnahmetest wird die Anwendung aus Sicht der Benutzerin oder des Benutzers getestet. Das Qualitätssicherungsteam verifiziert, ob alle Anwendungsfälle und Funktionen, welche in der Spezifikation definiert worden sind, vorhanden sind. Dafür muss eine lauffähige Anwendung vorhanden sein, um diese Tests durchführen zu können. Im Wasserfall-Vorgehensmodell kommt dieser Testansatz am Ende des Entwicklungszyklus. Es kommt dabei nicht selten vor, dass die Kundin oder der Kunde diese Tests manuell durchführt. Bei den agilen Vorgehensmodellen werden diese Tests nach jeder Iteration durchgeführt. Durch die kurzen Iterationszyklen können die Abnahmetests nicht mehr manuell durchgeführt werden. In diesem Fall kommen automatisierte Abnahmetests zum Einsatz. 

\SuperPar
Die große Herausforderung bei diesem Testansatz ist es, die Balance zwischen manuellen und automatisierten Tests zu finden.

\section{Verwendete Technologien}

Rayden basiert auf und verwendet eine Vielzahl von unterschiedlichen Technologien, Werkzeugen und Bibliotheken. Dieser Abschnitt gibt einen Einblick in die Technologien und erklärt, in welchen Bereichen diese im Rayden-System verwendet werden. Als Basis wird die Programmiersprache \enword{Java} und deren Laufzeitumgebung verwendet. Die Entscheidung für \enword{Java} ist essentiell für das Projekt, um eine große Anzahl an unterschiedlichen Test-Szenarien zu unterstützen. 

\SuperPar
Für die Umsetzung der Sprache wurden viele Bibliotheken und Werkzeuge aus dem Eclipse-Umfeld verwendet. Als Testtreiber-Bibliothek wird sowohl eine offene als auch eine kommerzielle Implementierung verwendet.

\subsection{Eclipse}

Eclipse \cite{Eclipse} ist eine Entwicklungsumgebung für eine Vielzahl an Programmiersprachen. Ursprünglich wurde Eclipse von IBM für die Sprache \enword{Java} entwickelt. Im Laufe der Zeit wurde Eclipse zu einer beliebten Entwicklerplattform und es wurden immer mehr Sprachen über \enword{Plug-ins} unterstützt. Auch für das Rayden-System soll ein solches \enword{Plug-in} entwickelt werden, um eine gute Unterstützung bei der Erstellung von Tests bieten zu können. 

\SuperPar
Neben der Entwicklungsumgebung ist Eclipse aber auch eine Plattform für die unterschiedlichsten Projekte geworden. Diese Projekte werden von der Eclipse Foundation \cite{EclipseFoundation} verwaltet und durch Partnerunternehmen und Freiwillige gepflegt. 

\SuperPar
Einige dieser Projekte werden in den nächsten Abschnitten separat vorgestellt.

\subsection{\enword{Eclipse Modeling Framework}}

Das \enword{Eclipse Modeling Framework} (EMF) \cite{EMF} ist ein Modellierungswerkzeug für \enword{Java}. EMF stellte eine Menge an Werkzeugen zur Erstellung, Verwaltung und Weiterverarbeitung zur Verfügung. Dazu gehört auch die Möglichkeit, aus diesen Modellen Code zu generieren. Eine Kernkomponente von EMF ist das ECore-Metamodell. Ein Metamodell ist die Schablone für ein spezifisches Modell. Aus einem ECore-Modell kann man mithilfe von Code-Generatoren eine \enword{Java}-Bibliothek generieren.

\SuperPar
Auf dieses Konzept baut auch das xText-Projekt auf, welches im nächsten Abschnitt vorgestellt wird. 

\subsection{xText}

Das xText-Projekt \cite{xtext} unterstützt das Erstellen von neuen Sprachen. Grundsätzlich ist xText ein Compiler-Generator der aus einer Grammatik einen lexikalischen und einen Syntax-Analysator generiert. Das Besondere an xText ist aber, dass man noch zusätzlich einen Eclipse-Editor für die Sprache bekommt. Der Editor bietet grundlegende Funktionen wie Syntax-\enword{Highlighting}, Fehler- und Validierungsmechanismen. Diese Funktionalität kann man nachträglich noch anpassen und weitere Funktionen hinzufügen. Ein Vorteil von xText ist, dass man den generierten Compiler auch außerhalb von Eclipse als eigenständige Anwendung verwenden kann. Somit kann der Aufwand zwei Compiler für seine Sprache zu warten eingespart werden. Der abstrakte Syntaxbaum einer Quelldatei wird im Compiler mit EMF umgesetzt. Das heißt, man bekommt einen vollständigen Syntax-Baum im Hauptspeicher, welchen man sehr einfach verarbeiten kann. Um die Verwendung noch zu vereinfachen, liegt für den Baum ein Metamodell in Form eines ECore-Modells vor. 

\subsection{Selenium}

Selenium \cite{Selenium} ist ein Open-Source-Projekt, um Webseiten automatisiert testen zu können. Die Bibliothek unterstützt eine Vielzahl an unterschiedlichen Browsern auf allen gängigen Betriebssystemen wie Windows, Linux, Mac und Google Android. Um die Browser ansprechen zu können, benötigt man einen speziellen Treiber. Dieser wird entweder als separate Anwendung aus- oder bereits mit dem Browser mitgeliefert.

\SuperPar
In der ersten Version hat Selenium auf eine proprietäre Programmierschnittstelle gesetzt. Seit der Version 2 setzt Selenium auf die standardisierte Programmierschnittstelle WebDriver \cite{WebDriver} vom W3C Konsortium. Der Vorteil von WebDriver ist, dass man eine einheitliche Programmierschnittstelle für die unterschiedlichsten Browser hat. Damit erzielt man Unabhängigkeit von einem spezifischen Browser. 

\subsection{Borland Silk Test}

Borland Silk Test \cite{SilkTest} ist eine kommerzielle Testsoftware für native wie auch Web-Anwendungen. Silk Test bietet Unterstützung für eine Vielzahl an unterschiedlichen Technologien. Unterstützt werden zum Beispiel die gängigen Browser wie Internet Explorer, Google Chrome und Mozilla Firefox. Neben Web-Technologien werden auch native Windows-, Adobe-Flex-, WPF- oder \enword{Java}-Anwendungen unterstützt. Seit kurzem werden auch Browser und Anwendungen  auf mobilen Geräten unterstützt. Der Vorteil von Silk Test gegenüber von Selenium ist, dass es einen \enword{X-Browser Support} gibt. Dabei kann man einen Test, welchen man mit dem Internet Explorer aufzeichnet, mit einem Mozilla-Firefox- oder dem Google-Chrome-Browser ausführen. Durch diese \enword{X-Browser}-Technologie entfällt die Wartung von Tests für die verschiedenen Browser. 

\chapter{Konzept}
\label{cha:Konzept}

\section{Ablauf eines Testprojekts}

TODO !!!

\subsection{Rollen in einem Testprojekt}

Testmanager, Entwickler, Tester

TODO !!!

\subsection{Der Testfall}

TODO !!!

\subsection{Komponenten- und Integrationstests für Testfälle}

TODO !!!

\subsection{Manuelle Abnahmetests für Testfälle}

TODO !!!

\subsection{Automatisieren von manuellen Abnahmetests}

TODO !!!

\subsection{Testdokumentation}

TODO !!!

\section{Die Entwicklung der Testautomatisierung}

\subsection{Record-Replay Testing}

\subsection{Functional Testing}

\subsection{Data-Driven-Testing}

\subsection{Keyword-Driven-Testing}

Erklären was KDT ist! 
Tests sind Daten.
Daten werden von einer Engine ausgewertet.

TODO !!!

\section{Robo-Framework: Ein Keyword-Driven-Testing-Framework}

TODO !!!

\section{Rayden}

Idee von Rayden. Weiterentwicklung von KDT.
Erklärung der Nachteile von KDT und wie man das mit Rayden lösen möchte.
Baustein Metapher.
TODO !!!


\chapter{Design von Rayden}
\label{cha:Design}

Im vorigen Kapitel wurde der Ablauf eines Testprojekts aufgezeigt und eine Einführung in das Thema \enword{Keyword-Driven Testing} gegeben. In diesem Kapitel wird das Rayden-System detailliert erklärt. Zu Beginn werden die Designziele der Sprache Rayden erklärt. Die Sprache Rayden ist eine domänenspezifische Sprache, welche einige Eigenheiten und Überraschungen enthält. In den weiteren Abschnitten wird der Aufbau des Rayden-Systems erklärt und auf die technischen Details eingegangen. Am Ende dieses Kapitels wird noch die Integration in die \enword{Java-Scripting-API} \cite{JavaScriptApi} beschrieben. Das Rayden-System bietet die Möglichkeit einen Test in einer \enword{Java}-Anwendung über das \enword{Java-Scripting-API} auszuführen.

%%------------------------------------------------------------------------------------------------------

\section{Designziele von Rayden}

Das primäre Designziel von Rayden ist Offenheit. Rayden soll im gesamten Testprozess einsetzbar sein, darf aber die involvierten Personen nicht überfordern. Um dieses Ziel zu erreichen, setzt das Rayden-System auf mehreren Ebenen an.

\SuperPar
Die domänenspezifische Sprache von Rayden ist speziell für Personen im Testbereich ausgelegt. Das wichtigste Ziel der Sprache ist Einfachheit. Die Sprache soll Personen ohne Programmierkenntnisse in die Lage versetzen, Tests in dieser Sprache lesen und bearbeiten zu können. Die Sprache Rayden ist daher stark an der natürlichen Sprache angelehnt, um den Einstieg zu erleichtern. Ein anderes Ziel bei dem Sprachdesign ist die Flexibilität der Sprache. Der Testprozess setzt sich aus einer Vielzahl an unterschiedlichen Aufgaben zusammen. Um möglichst alle Aufgaben mit dieser Sprache abdecken zu können, wird eine hohe Flexibilität benötigt. 

\SuperPar
Abgesehen von einer geeigneten Sprache gibt es noch weitere wichtige Ziele für das Rayden-System. Rayden muss plattformunabhängig sein, um viele Anwendungsszenarien abdecken zu können. Aus diesem Grund wird die Programmiersprache \enword{Java} für die Entwicklung des Rayden-Systems gewählt. Der Interpreter für Rayden läuft ebenfalls auf der virtuellen \enword{Java}-Maschine (JVM).

\SuperPar
Die Einbindung von externen Testtreiber-Bibliotheken wird durch eine offene Schnittstelle ermöglicht. Dadurch können Rayden-Tests für die unterschiedlichsten Anwendungsszenarien entwickelt werden. Rayden kann somit für das jeweilige Projekt und die beteiligten Personen angepasst werden. 

%%------------------------------------------------------------------------------------------------------

\section{Aufbau des Rayden-Systems}

In diesem Abschnitt wird der Aufbau des Rayden-Systems von zwei Blickwinkeln aus beleuchtet. Zuerst wird der konzeptionelle Aufbau erklärt. Dabei wird darauf eingegangen, wie die einzelnen Konzeptebenen miteinander kommunizieren und welche Person für die jeweilige Ebene verantwortlich ist. Im zweiten Teil wird die technische Architektur des Rayden-Systems erläutert. Dazu werden die Komponenten und ihre Beziehungen überblicksweise erklärt. Eine ausführliche Beschreibung findet man in den Abschnitten \ref{cha:KonzeptAufbau} und \ref{cha:TechArch}.

\subsection{Konzeptioneller Aufbau}
\label{cha:KonzeptAufbau}

Wie schon in vorigen Abschnitten erwähnt, ist Rayden an das Konzept von \enword{Keyword-Driven Testing} angelehnt. Bevor der Aufbau von Rayden beschrieben wird, wird der konzeptionelle Aufbau von \enword{Keyword-Driven Testing} erläutert. 

\begin{figure}[h]
\centering
\includegraphics[width=0.7\textwidth]{KDT-Architektur.png}
\caption{Aufbau von \enword{Keyword-Driven Testing}}
\label{fig:kdt-arch}
\end{figure}

\SuperPar
Ein \enword{Keyword-Driven}-Test besteht aus einer Sequenz von \enword{Keywords}. Diese \enword{Keywords} können wiederum aus einer Sequenz von \enword{Keywords} bestehen oder mit einem Codestück verbunden sein. Die \enword{Keywords} werden in drei Kategorien, wie in Abbildung \ref{fig:kdt-arch} dargestellt, aufgeteilt. Die \enword{High-Level Keywords} repräsentieren einen Testfall mit einer detaillierten Beschreibung. Diese Gruppe von \enword{Keywords} wird typischerweise von einer Person aus der Fachabteilung oder von einer Testmanagerin oder einem Testmanager erstellt. Dabei wird aber nur der Rumpf des \enword{Keywords} erstellt. Die Implementierung wird erst in der nächsten Phase hinzugefügt. Diese \enword{High-Level Keywords} bilden die Grundlage für die Erstellung der Tests. In der weiteren Phase werden diese \enword{Keywords} von Testerinnen und Testern implementiert.

\SuperPar
Die \enword{High-Level Keywords} bestehen typischerweise aus einer Sequenz von \enword{Mid-Level Keywords}. Diese Sequenz wird in der zweiten Phase erstellt. Normalerweise finden sich \enword{Mid-Level Keywords} in dieser Sequenz, es können aber auch \enword{Low-Level Keywords} verwendet werden. Die verwendeten \enword{Mid-Level Keywords} bestehen wiederum aus einer Sequenz von \enword{Mid-Level Keywords} und \enword{Low-Level Keywords}. Technisch gesehen gibt es keinen Unterschied zwischen \enword{High-Level Keywords} und \enword{Mid-Level Keywords}. Der Unterschied besteht nur in der Art der Verwendung. \enword{High-Level Keywords} beschreiben genau einen Anwendungsfall, der getestet werden soll. Im Gegenteil zu \enword{Mid-Level Keywords} wird hier kein Wert auf Wiederverwendung gelegt.

\SuperPar
In der letzten Phase werden \enword{Low-Level Keywords} mit Code verbunden. Der Code kann prinzipiell in jeder Programmiersprache geschrieben sein. Die Wahl der Programmiersprache hängt von dem verwendeten \enword{Keyword-Driven Framework} ab. In diesen \enword{Keyword-Driven Frameworks} werden häufig Skript-Sprachen verwendet. Der Vorteil von Skript-Sprachen liegt darin, dass der Code für die Ausführung des Tests nicht kompiliert werden muss.

\begin{figure}[h]
\centering
\includegraphics[width=0.7\textwidth]{Rayden-Architektur.png}
\caption{Aufbau von Rayden}
\label{fig:rayden-arch}
\end{figure}

\SuperPar
Im Gegensatz zu \enword{Keyword-Driven Testing} unterteilt das Rayden-System die \enword{Keywords} in mehr Gruppen, wie in Abbildung \ref{fig:rayden-arch} dargestellt. Die zusätzlichen Gruppen bieten einen bessere Strukturierung und geben eine klare Richtung vor, wie ein Rayden-Test-Projekt aufgebaut werden soll.    

\SuperPar
Rayden führt eine klare Trennung bei \enword{Low-Level Keywords} ein. Diese \enword{Keywords}, welche mit einem Codestück verbunden sind, werden in \enword{Library-} und \enword{Bridge-Keywords} unterteilt. \enword{Library-Keywords} stellen grundlegende Funktionen bereit, welche unabhängig von einem speziellen Anwendungsfall sind. Als Beispiel kann man sich eine \enword{For}- oder \enword{Print-Keyword} vorstellen. Im Gegensatz dazu sind \enword{Bridge-Keywords} speziell für eine Anwendungstechnologie angepasst, wie zum Beispiel \enword{Open Browser} oder \enword{Navigate To} für Web-Anwendungen. 

\SuperPar
Auch bei den \enword{High-Level Keywords} bietet Rayden eine größere Vielfalt. Grob werden diese in Test-Suiten und Testfälle unterteilt. Die Test-Suite dient als Gruppierungselement für Testfälle, um diese gemeinsam ausführen zu können. Bei der Definition von Testfällen können diese mit unterschiedlichen Testtypen angelegt werden. Eine nähere Beschreibung findet sich im Abschnitt \ref{cha:KeywordTypes}.

\SuperPar
Zum Abschluss ist noch auf das \enword{Object Repository} hinzuweisen. Diese Komponente verwaltet Testobjekte. Testobjekte können für die Beschreibung von Benutzeroberflächen-Komponenten wie Schaltflächen oder Eingabefelder verwendet werden. Dafür wird für jedes Testobjekt ein Bezeichner definiert, mit welchem man die Komponente in der Benutzeroberfläche finden kann. Das ist im Fall einer Web-Anwendung ein \enword{XPath}- oder ein \enword{CSS}-Ausdruck. Das \enword{Object Repository} sorgt somit für eine klare Trennung zwischen Test- und Benutzeroberflächen-Beschreibung. Diese Trennung erhöht die Wiederverwendbarkeit von \enword{Keywords} und reduziert den Wartungsaufwand bei Änderungen an der Benutzerschnittstelle.

\subsection{Technische Architektur}
\label{cha:TechArch}

Die technische Basis für das Rayden-System ist die \enword{Java}-Plattform. Auf der Entwicklungsseite wird \enword{Java} als Programmiersprache für das gesamte Rayden-System verwendet, auf der Ausführungsseite läuft das Rayden-System auf der virtuellen \enword{Java}-Maschine (JVM). Außerdem bietet Rayden die Möglichkeit, dass man einen Rayden-Test über das \enword{Java Scripting API} \cite{JavaScriptApi} ausführen kann. 

\begin{figure}[h]
\centering
\includegraphics[width=0.9\textwidth]{Rayden-Tech-Architektur.png}
\caption{Rayden-Architektur}
\label{fig:rayden-tech-arch}
\end{figure}

\SuperPar
Abbildung \ref{fig:rayden-tech-arch} zeigt alle Komponenten des Rayden-Kernsystems in Grün. Diese Komponenten bilden die Grundlage dafür, einen Rayden-Test ausführen zu können. Als Basis dieser Komponenten sieht man in Blau die virtuelle \enword{Java}-Maschine (JVM). Die externen \enword{Bridge}-Implementierungen werden in Rot dargestellt. Diese Komponenten stellen eine Verbindung zwischen dem Test-Treiber und der Rayden-\enword{Runtime} her und werden über die \enword{Bridge}-Schnittstelle hergestellt. Im oberen Abschnitt der Abbildung \ref{fig:rayden-tech-arch} sieht man das \enword{Java Scripting API}, über welches man Rayden-Tests ausführen kann.

\SuperPar
Nachfolgend werden die einzelnen Komponenten der Rayden-Architektur beschrieben, um einen groben Überblick über die Funktionsweise von Rayden zu geben.\\

\begin{itemize}

\item \textbf{Runtime}

Die \enword{Runtime} ist der Einstiegspunkt für die Ausführung von Rayden-Tests. Dazu enthält diese Komponente die Implementierung für die \enword{Java Scripting API}. Wird ein Test ausgeführt, werden zuerst alle Projektressourcen in die \enword{Runtime} geladen. Das Projektverzeichnis kann man über einen Kontextparameter setzen. Falls dieser nicht gesetzt ist, wird das aktuelle Verzeichnis verwendet. Für das Laden der Ressourcen wird die Compiler-Komponente verwendet. Die \enword{Runtime} baut bei diesem Lesevorgang eine \enword{Lookup}-Tabelle für alle \enword{Keywords} auf. Diese Tabelle wird für einen schnellen Zugriff im Interpreter benötigt. Der Rayden-Test wird mithilfe des Interpreters ausgeführt. Das Ergebnis des Tests wird als Resultat über die \enword{Java Scripting API} zurückgegeben.\\

\item \textbf{Compiler}

Der Compiler für die Rayden-Sprache wird mit dem Compiler-Werkzeug xText \cite{xtext} realisiert. Von dem generierten Compiler wird für die Ausführungseinheit nur der lexikalische und syntaktische Analysator verwendet. Der Eclipse-Editor wird für das Rayden-System nicht benötigt. Das Resultat der Compiler-Komponente ist ein EMF-Modell des Tests. Die \enword{Runtime}-Komponente verwaltet die Modelle und stellt diese dem Interpreter bei Bedarf zur Verfügung.\\

\item \textbf{Interpreter}

Der Interpreter ist die wichtigste Komponente im Rayden-System. Der Interpreter ist dafür verantwortlich, dass die Rayden-Tests ausgeführt werden können. Zum Starten des Interpreters wird der Aufruf eines Test-\enword{Keywords} übergeben. Dieses \enword{Keyword} wird auf den leeren \enword{Stack} geladen. Der Test wird mithilfe einer \enword{Stack}-Maschine \cite{StackMachine} abgearbeitet. Bei jedem Aufruf eines \enword{Keywords} wird die \enword{Keyword}-Implementierung auf den \enword{Stack} geladen. Zusätzlich wird für jeden neuen \enword{Keyword}-Aufruf ein neuer Gültigkeitsbereich (\enword{Scope}) angelegt. \\
\\
Der Gültigkeitsbereich ist für die Verwaltung der Parameter und Variablen zuständig. Eine Besonderheit in Rayden ist, dass Gültigkeitsbereiche Zugriff auf andere Gültigkeitsbereiche haben. Eine detaillierte Beschreibung dazu findet sich im Abschnitt \ref{cha:KeywordScope}. Die Auswertung von Ausdrücken wird in einem separaten Teil des Interpreters vorgenommen. Für die Auswertung werden der aktuelle Gültigkeitsbereich und der Ausschnitt aus dem Modell an die Evaluierungskomponente übergeben. Das Ergebnis des Ausdrucks wird wieder auf den \enword{Stack} gelegt. Ruft die \enword{Stack}-Maschine ein \enword{Scripted-Keyword} (Beschreibung in Abschnitt \ref{cha:Keyword}) auf, wird entweder der dazugehörige Code ausgeführt oder es wird der Aufruf an die \enword{Bridge}-Schnittstelle weitergeleitet.\\

\item \textbf{Bridge-Schnittstelle}

Die \enword{Bridge}-Schnittstelle ist für die Anbindung von Test-Treibern verantwortlich. Um einen Test-Treiber verwenden zu können, muss eine Rayden-\enword{Bridge} implementiert werden. Die \enword{Bridge} mit der Schnittstelle bildet die Verbindung zwischen der Rayden-\enword{Runtime} und dem Test-Treiber.\\

\item \textbf{Object Repository}

Das \enword{Object Repository} verwaltet Testobjekte, welche von \enword{Keywords} verwendet werden können. Die Testobjekte werden in einer Baumstruktur verwaltet. Die wichtigste Eigenschaft eines Testobjekts ist der Bezeichner (\enword{Locator}). Mit dem Bezeichner kann die Benutzerschnittstellen-Komponente identifiziert werden. Das Konzept ist an der Idee von \enword{Page-Object-Pattern} \cite{PageObject} angelehnt. \\

\end{itemize}


%%------------------------------------------------------------------------------------------------------

\section{Sprache von Rayden}

Als Inspiration und Basis für die Sprache dient das Konzept von \enword{Keyword-Driven Testing}. Das Grundprinzip hinter \enword{Keyword-Driven Testing} ist die Verwendung von \enword{Keywords}. Ein \enword{Keyword} besteht aus einer Sequenz von anderen \enword{Keywords} oder ist mit einem Codestück verbunden. Einen \enword{Keyword-Driven}-Test kann man sich auch als gerichteten Graph vorstellen, in dem die Knoten die \enword{Keywords} repräsentieren und die Kanten die Abhängigkeit zwischen den \enword{Keywords} beschreiben, wie in Abbildung \ref{fig:test-graph} dargestellt.

\begin{figure}[h]
\centering
\includegraphics[width=0.8\textwidth]{Rayden-Testbaum.png}
\caption{Graph-Repräsentation eines Tests}
\label{fig:test-graph}
\end{figure}

\SuperPar
Der gelbe Knoten repräsentiert ein \enword{High-Level Keyword}. Von diesem Knoten aus werden über gerichtete Kanten die Beziehungen zu den \enword{Mid-Level Keywords} in Grün definiert. Man sieht, dass die \enword{Mid-Level Keywords} entweder wieder in Beziehungen zu anderen \enword{Mid-Level Keywords} stehen oder \enword{Low-Level Keywords} referenzieren. Die \enword{Low-Level Keywords} werden in Blau dargestellt. Bei dem Graph handelt es um einen gerichteten azyklischen Graph (DAG).

\SuperPar
Die Rayden-Sprache setzt auch auf dieses Konzept von \enword{Keywords}. Im Unterschied zu \enword{Keyword-Driven Testing} setzt Rayden auf eine größere Vielfalt an unterschiedlichen \enword{Keywords}, welche im nächsten Abschnitt \ref{cha:Keyword} detailliert beschrieben werden. Ein weiterer Unterschied ist die Benennung von \enword{Keywords}. Normalerweise besteht der Name eines \enword{Keywords} nur aus einem Wort, damit die Verarbeitung der Tests für den Compiler erleichtert wird. In der Rayden-Sprache wird ein großes Augenmerk darauf gelegt, dass man nicht nur auf ein Wort beschränkt ist, sondern auch ganze Sätze als Namen verwenden kann. Diese Eigenschaft ist sehr nützlich, um die Testfälle wie in einer natürlichen Sprache beschreiben zu können. Der Vorteil ist, dass man somit ohne weiteren Aufwand eine ordentliche Dokumentation des Tests bekommt.

\SuperPar
Eine andere interessante Eigenschaft der Sprache ist, dass in der Sprache keine Sprung-Operationen enthalten sind. Die Konsequenz daraus ist, dass es in der Sprache auch keine Schleifen- oder Verzweigungs-Konstrukte enthalten sind. Die einzige Möglichkeit, um ähnliche Konstrukte zur Verfügung zu stellen, sind \enword{Scripted Compound Keywords}, wobei man bei diesem Metatyp von \enword{Keyword} auch nur entscheiden kann, ob eine Sequenz von \enword{Keywords} ausgeführt werden soll. Das Konzept der Metatypen wird im Abschnitt \ref{cha:Keyword} beschrieben. 

\SuperPar
Da Sprung-Operationen vermieden werden und die Sprache blockstrukturiert ist, gewinnt man die Fähigkeit, Tests visuell darstellen zu können. Diese Fähigkeit ist hilfreich, um eine bessere Unterstützung und einen leichteren Einstieg in die Sprache zu ermöglichen. Das ist vor allem von Vorteil, wenn Personen aus einer Fachabteilung nur unregelmäßig damit arbeiten müssen. 

%%------------------------------------------------------------------------------------------------------

\section{\enword{Keywords} von Rayden}
\label{cha:Keyword}

Das \enword{Keyword} ist die Schlüsselkomponente der Sprache Rayden. In diesem Abschnitt werden die unterschiedlichen Typen und Metatypen erklärt und gezeigt, wofür diese verwendet werden können. Am Anfang werden die vier Metatypen von \enword{Keywords} erklärt. Die Metatypen sind die Basis für den Funktionsumfang der Sprache. Ferner werden die unterstützten Typen beschrieben und wofür diese verwendet werden können. Als Abschluss werden noch Themen wie Sichtbarkeit, Benennung und Parameter erläutert.

%%------------------------------------------------------------------------------------------------------

\subsection{Metatypen}

Der Metatyp definiert die Funktionsweise eines \enword{Keywords}. Rayden unterscheidet zwischen vier Metatypen, wobei einer dieser Metatypen nur eine Kurzform ist.

%%------------------------------------------------------------------------------------------------------

\subsection{Metatyp: \enword{Compound Keyword}}

Das \enword{Compound Keyword} ist die einfachste Variante eines \enword{Keywords}. Bei einem \enword{Compound Keyword} wird eine Sequenz von \enword{Keywords} zu einem neuen \enword{Keyword} zusammengefasst. Der Beispiel-Code \ref{prog:compoundKeyword} zeigt die Verwendung des \enword{Compound Keywords} \enword{Anmelden an der PetClinic Anwendung}. In dem Beispiel kann man gut sehen, dass dieser Metatyp hauptsächlich für die Strukturierung von Tests verwendet wird. Ein mögliches Vorgehen kann dabei sein, dass man einen Testfall immer weiter und weiter in \enword{Compound Keywords} zerlegt, bis man am Ende die Aufgabe auf einzelne Aktionen heruntergebrochen hat. Für diese Aktionen werden dann \enword{Scripted Keywords} verwendet wie im Code-Beispiel die beiden \enword{Keywords} \enword{Type Text} und \enword{Click Left}.

\begin{program}
\begin{JavaCode}
keyword Anmelden an der PetClinic Anwendung {
	'''Man meldet sich bei der Anwendung PetClinic mit den definierten 
	   Daten an. Wenn das Keyword erfolgreich ausgeführt wurde, 
	   befindet man sich auf der Hauptseite der Webanwendung.'''
	
	parameter in username as string
	parameter in password as string
	
	Type Text(@PetClinic.LoginPage.Username, username)
	Type Text(@PetClinic.LoginPage.Password, password)
	
	Click Left(@PetClinic.LoginPage.LoginButton)
}
\end{JavaCode}
\caption{Das Beispiel zeigt das \enword{Compound Keyword} \enword{Anmelden an der PetClinic Anwendung}}
\label{prog:compoundKeyword}
\end{program}

\SuperPar
Ein klares Ziel bei der Erstellung von \enword{Compound Keywords} ist die Wiederverwendung. Ein \enword{Compound Keyword} soll als eine logische Einheit aufgebaut werden, sodass man diese auch wieder für andere Tests verwenden kann. 

%%------------------------------------------------------------------------------------------------------

\subsection{Metatyp: \enword{Inline Keyword}}

Der Metatyp \enword{Inline Keyword} ist eine Kurzform des \enword{Compound Keywords}. Dabei kann man in einem \enword{Compound Keyword} ein neues \enword{Keyword} erstellen. Daher kommt auch der Name \enword{Inline Keyword}, weil es innerhalb eines anderen \enword{Keywords} erstellt wird. Im Beispiel \ref{prog:inlineKeyword} wird im \enword{Keyword} \enword{Anmelden an der PetClinic Anwendung} das \enword{Inline Keyword} \enword{Besitzer anlegen} definiert. Es werden alle Schritte zum Anlegen eines neuen Besitzers zusammengefasst. Der Anwendungsfall dieses Metatyps ist wiederum die Strukturierung, aber in diesem Fall innerhalb eines \enword{Keywords}. 

\begin{program}
\begin{JavaCode}
testcase Anlegen eines neuen Besitzers {
	'''Der Testfall überprüft den Anwendungsfall um einen 
	   neuen Besitzer anlegen zu können.'''
	
	Anmelden an der PetClinic Anwendung ("max", "secret")
	
	Besitzer anlegen {
		Oeffnen der Besitzerseite		
		Neuen Besitzer in der Anwendung anlegen("Huber", "Mayr")
		Daten von Besitzer ueberpruefen
	}
	
	Abmelden von der Anwendung
}
\end{JavaCode}
\caption{Beispiel eines \enword{Inline Keywords}}
\label{prog:inlineKeyword}
\end{program}

\SuperPar
Der Nachteil bei dieser Variante ist, dass man dieses \enword{Keyword} nicht wiederverwenden kann. Das \enword{Inline Keyword} ist nur innerhalb des \enword{Compound Keywords} bekannt.

%%------------------------------------------------------------------------------------------------------

\subsection{Metatyp: \enword{Scripted Keyword}}

Das \enword{Scripted Keyword} ist der einfachere Metatyp, mit welchem man Code an ein \enword{Keyword} binden kann. Ein \enword{Scripted Keyword} wird wie ein \enword{Compound Keyword} definiert. Im Unterschied dazu besitzt das \enword{Scripted Keyword} keine Sequenz von \enword{Keywords}, sondern einen Hinweis auf die Implementierung. Im Beispiel-Code \ref{prog:scriptedKeyword} sieht man eine Variante mit einer \enword{Java}-Implementierung. Die Anweisung \enword{implemented in java} definiert die Implementierungssprache. Nach dem Pfeil folgt ein Bezeichner, welcher die Implementierung referenziert. Im Fall von \enword{Java} wird der vollständige Name der Klasse verwendet.

\begin{program}
\begin{JavaCode}
keyword Print {
	'''Der Parameter 'text' wird in den Test-Report geschrieben.'''
	
	parameter text
	implemented in java -> "com.github.thomasfischl.rayden.runtime.keywords.impl.PrintKeyword"
}
\end{JavaCode}
\caption{Rayden: Beispiel \enword{Scripted Keyword}}
\label{prog:scriptedKeyword}
\end{program}

\SuperPar
Um die \enword{Java}-Klasse als \enword{Keyword}-Implementierung verwenden zu können, muss die Klasse das \enword{Interface} \enword{ScriptedKeyword} implementieren. Das \enword{Interface} hat nur die Methode \enword{execute}. Kommt die \enword{Stack}-Maschine zu einem Aufruf eines \enword{Scripted Keywords}, wird ein neues Objekt der \enword{Keyword}-Implementierung über den \enword{Java-Reflection}-Mechanismus angelegt. Von dem Objekt wird dann die Methode \enword{execute} mit dem Namen des aktuellen \enword{Keywords}, dem Gültigkeitsbereich und einem \enword{Reporter}-Objekt aufgerufen. Auf die Parameter des \enword{Keywords} kann man über den Gültigkeitsbereich zugreifen, wie man im Beispiel-Code \ref{prog:scriptedKeywordImpl} sehen kann. Als Ergebnis liefert die Methode ein \enword{KeywordResult}-Objekt. Dieses Objekt signalisiert der \enword{Stack}-Maschine, ob das \enword{Keyword} erfolgreich ausgeführt wurde.   

\begin{program}
\begin{JavaCode}
public class PrintKeyword implements ScriptedKeyword {

	@Override
	public KeywordResult execute(String keyword, 
			IKeywordScope scope, IRaydenReporter reporter) {
		reporter.log(scope.getVariable("text").toString());
		return new KeywordResult(true);
	}
}
\end{JavaCode}
\caption{Rayden: \enword{Java}-Implementierung des \enword{Print Keywords}}
\label{prog:scriptedKeywordImpl}
\end{program}

\SuperPar
Der Parameter \enword{keyword} bei der Methode \enword{execute} wird benötigt, weil es in Rayden möglich ist, eine Implementierung an mehrere \enword{Keyword}-Definitionen zu binden. Mit diesem Parameter kann man den Namen der aktuellen \enword{Keyword}-Definition abfragen.

\SuperPar
Über das \enword{Reporter}-Objekt kann man Einträge in den Test-Report hinzufügen. Die Instanz bietet unterschiedliche Granularitätsstufen für Nachrichten. Es werden spezielle Methoden für die Stufen Fehler, Warnung und Information angeboten. Diese Nachrichten können in der Folge von den jeweiligen \enword{Reporter}-Implementierungen unterschiedlich behandelt werden. 

%%------------------------------------------------------------------------------------------------------

\subsection{Metatyp: \enword{Scripted Compound Keyword}}

Das \enword{Scripted Compound Keyword} ist die komplizierteste Variante der vier Metatypen, jedoch ist diese Variante essentiell für die Flexibilität der Sprache. Mit dem Konzept von \enword{Scripted Compound Keywords} ist eine Entwicklerin oder ein Entwickler in der Lage, die Sprache um Kontrollstrukturen zu erweitern. Dafür werden die Eigenschaften von \enword{Compound Keywords} und \enword{Scripted Keywords} kombiniert. 

\begin{program}
\begin{JavaCode}
keyword IF { 
	parameter in condition as boolean
	implemented in java -> "com.github.thomasfischl.rayden.runtime.keywords.impl.IfKeyword"
}
\end{JavaCode}
\caption{Beispiel eines \enword{Scripted Compound Keywords}}
\label{prog:ifKeyword}
\end{program}

\SuperPar
Das \enword{Scripted Compound Keyword} ist mit einem Codestück verbunden und hat zusätzlich noch eine \enword{Keyword}-Liste. In der Implementierung hat man die Möglichkeit, die Ausführung der \enword{Keyword}-Liste zu steuern. Man kann damit eine bedingte bzw. mehrmalige Ausführung der Liste realisieren. Es ist aber zu beachten, dass man die Liste nur als Ganzes steuern kann. Eine teilweise Ausführung der Liste ist nicht möglich.

\SuperPar
Das Beispiel \ref{prog:ifKeyword} zeigt die Definition für ein \enword{IF Keyword}. Dabei werden wie bei einem \enword{Scripted Keyword} die Programmiersprache und der Bezeichner definiert. Im Fall von einem \enword{Scripted Compound Keyword} muss die Klasse das \enword{Interface} \enword{ScriptedCompoundKeyword} implementieren. Dieses \enword{Interface} ist deutlich schwieriger zu implementieren, wie man im Beispiel-Code \ref{prog:ifKeywordImpl} sehen kann.

\begin{program}
\begin{JavaCode}
public class IfKeyword implements ScriptedCompoundKeyword {

  private IKeywordScope scope;

  @Override
  public void initializeKeyword(String keyword, IKeywordScope scope, IRaydenReporter reporter) {
    this.scope = scope;
  }

  @Override
  public boolean executeBefore() {
    return scope.getVariableAsBoolean("condition");
  }

  @Override
  public boolean executeAfter() {
    return false;
  }
  
  @Override
  public KeywordResult finalizeKeyword() {
    return new KeywordResult(true);
  }
}
\end{JavaCode}
\caption{\enword{Java}-Implementierung des \enword{IF Keywords}}
\label{prog:ifKeywordImpl}
\end{program}

\SuperPar
Das \enword{Interface} enthält für jede der vier Phasen eines \enword{Scripted Compound Keywords} eine Methode, in der man die Ausführung steuern kann. \\

\begin{itemize}

\item \textbf{Phase 1: Initialisierung (\enword{initializeKeyword})}\\
In der Initialisierungsphase wird der aktuelle Zustand von der \enword{Stack}-Maschine an die \enword{Keyword}-Implementierung übergeben. Falls die Informationen für die Ausführung benötigt werden, können diese im Objekt gespeichert werden. Eine Instanz der \enword{Keyword}-Implementierung wird genau für eine Ausführung verwendet. Das heißt, man kann keinen globalen Zustand für zukünftige Ausführungen speichern. Falls man diese Funktionalität benötigt, muss man diese Daten in Klassenvariablen speichern.\\

\item \textbf{Phase 2: Beginn der Auswertung (\enword{executeBefore})}\\
Die Ausführung der \enword{Keyword}-Liste kann in dieser Phase beeinflusst werden. Diese Methode wird vor jeder Auswertung der \enword{Keyword}-Liste aufgerufen. Wenn diese Methode \enword{false} liefert, wird die Liste nicht ausgewertet und es wird zur Phase 4 gesprungen.\\

\item \textbf{Phase 3: Beendigung der Auswertung (\enword{executeAfter})}\\
Nach der Ausführung der \enword{Keyword}-Liste wird die Methode \enword{executeAfter} aufgerufen. In dieser Phase wird entschieden, ob die Liste ein weiteres Mal ausgeführt werden soll. Wenn die Methode in dieser Phase \enword{true} liefert, wird die Ausführung bei der zweiten Phase fortgesetzt. Ansonsten wird die vierte Phase ausgeführt.\\

\item \textbf{Phase 4: Beendigung des Keywords (\enword{finalizeKeyword})}\\
In der letzten Phase können noch Abschlussarbeiten vorgenommen werden, wie beispielsweise die Berechnung des Status für das \enword{Keyword}. Der Status signalisiert, ob die Ausführung erfolgreich war oder nicht. Diese Funktionalität kann für Validierungen verwendet werden.

\end{itemize}

\SuperPar
Die Verwendung eines \enword{Scripted Compound Keyword} sieht wie ein \enword{Inline Keyword} aus. Der große Unterschied ist, dass ein \enword{Inline Keyword} keine Parametersignatur im Gegensatz zu einem \enword{Scripted Compound Keyword} hat. Ein Beispiel für die Verwendung findet man im Code-Ausschnitt \ref{prog:ifKeywordUsage}.

\begin{program}
\begin{JavaCode}
keyword If Keyword Bespiel {
	If (a == 1) {
		Print("Condition is true")
	}
	If (test == "b") {
		Print("Condition is false")
	}
}
\end{JavaCode}
\caption{Verwendung des \enword{IF Keywords}}
\label{prog:ifKeywordUsage}
\end{program}

%%------------------------------------------------------------------------------------------------------
\subsection{Typen}
\label{cha:KeywordTypes}

Neben den Metatypen für \enword{Keywords} gibt es in Rayden auch unterschiedliche Typen von \enword{Keywords}. Die Typen liefern keine zusätzliche Funktionalität für die Sprache, sondern dienen als Strukturierungselement für Testprojekte. Mithilfe der Typen kann man Testfälle unterscheiden und eine klare Zuordnung zu einer Testmethode treffen. Durch die Typen wird es auch möglich, eine Aussage über die Verteilung der Testmethoden in einem Testprojekt treffen zu können. 

\SuperPar
Rayden unterstützt die folgenden \enword{Keyword}-Typen:

\begin{itemize}
\item Test-Suite (\enword{TestSuite}),
\item Testfall (\enword{TestCase}),
\item Komponententest (\enword{UnitTest}),
\item Integrationstest (\enword{IntegrationTest}),
\item Schnittstellentest (\enword{APITest}),
\item Automatisierter Abnahme-Test (\enword{AUTest}) und
\item Manueller Abnahme-Test (\enword{MAUTest}).
\end{itemize}

\SuperPar
In Rayden ist es aber nicht zwingend notwendig, diese Typen zu verwenden. Man kann statt den Typen einfach das Schlüsselwort \enword{keyword} verwenden. Damit verliert man aber die Auswertungsmöglichkeit in einem Testprojekt. 

%%------------------------------------------------------------------------------------------------------
\subsection{Gültigkeitsbereich}
\label{cha:KeywordScope}

In Rayden wird für jeden Aufruf eines \enword{Keywords} ein neuer Gültigkeitsbereich angelegt. In diesem Gültigkeitsbereich befinden sich alle Parameter, welche für das \enword{Keyword} definiert sind. Nachdem Parameter in einem Gültigkeitsbereich definiert wurden, verhalten sich diese gleich wie Variablen. Variablen können von einem \enword{Keyword} in einem Gültigkeitsbereich mit einem Wert belegt werden. Der Wert einer Variablen kann entweder in einem Ausdruck oder in einem \enword{Keyword} verwendet werden. In Rayden müssen Variablen nicht deklariert werden. Sobald das erste Mal eine Variable mit einem Wert belegt wurde, ist diese im Gültigkeitsbereich vorhanden.

\begin{figure}[h]
\centering
\includegraphics[width=0.5\textwidth]{Rayden-Scope.png}
\caption{Gültigkeitsbereiche in einem Rayden-Test}
\label{fig:rayden-scope}
\end{figure}

\SuperPar
In Rayden gibt es aber noch eine Besonderheit in Bezug auf Gültigkeitsbereiche. Rayden verwendet für die Gültigkeitsbereiche das Konzept von \enword{Dynamic Scoping}. Bei einem Aufruf eines \enword{Keywords} wird der Gültigkeitsbereich mit den Parametern angelegt. Das Besondere ist, dass der neue Gültigkeitsbereich eine Referenz auf den alten Gültigkeitsbereich hat. Das Resultat ist, dass jeder Kind-Gültigkeitsbereich Zugriff auf den Eltern-Gültigkeitsbereich hat.

\SuperPar
Ein Beispiel dazu sieht man in der Abbildung \ref{fig:rayden-scope}. Beim Starten eines Tests wird der Gültigkeitsbereich G1 angelegt. Auf diesen Gültigkeitsbereich haben später alle anderen Gültigkeitsbereiche Zugriff. Daher eignet sich dieser Gültigkeitsbereich gut für globale Variablen.

\SuperPar
In der Abbildung sieht man weiter, dass jeder neue Gültigkeitsbereich eine Beziehung zu einem Eltern-Gültigkeitsbereich hat. Der Pfeil von einem Kind- zu einem Eltern-Gültigkeitsbereich mag am Anfang ungewöhnlich wirken. Der Pfeil erklärt sich aber damit, dass man in Rayden \enword{Out-} und \enword{InOut}-Parameter definieren kann. Damit können Variablen aus G2 an G1 übertragen werden. Eine detaillierte Beschreibung zu den Parametern findet man im Abschnitt \ref{cha:Parameter}.

\SuperPar
Der Vorteil von vererbten Gültigkeitsbereichen ist, dass nicht alle Variablen übergeben werden müssen, welche bei einem Test zahlreich vorkommen können. Die Parameter bieten die Möglichkeit für eine explizite Definition von Variablen. Das wird verwendet, um sicherzustellen, dass eine Variable definitiv zur Verfügung steht bzw. erleichtert auch die Verwendung eines \enword{Keywords}.

%%------------------------------------------------------------------------------------------------------
%\subsection{Benennung}
%\todo 

%%------------------------------------------------------------------------------------------------------
\subsection{Parameter}
\label{cha:Parameter}

\enword{Keywords} unterstützen das Definieren von Parametern für eine einfachere Verwendung. Grundsätzlich werden Parameter in Rayden nicht zwingend benötigt, da Rayden das Konzept von vererbten Gültigkeitsbereichen verwendet. Parameter ermöglichen jedoch eine explizite Schnittstelle für \enword{Keywords}. 

\begin{program}
\begin{JavaCode}
keyword Parameter Beispiel {
	
	parameter in    parm1
	parameter in    parm2 as string
	parameter out   param3 as boolean
	parameter inout param4 as number
	
	Test1	
}
\end{JavaCode}
\caption{Verwendung von Parametern}
\label{prog:parameter}
\end{program}


\SuperPar
Rayden unterstützt sowohl typisierte als auch untypisierte Parameter. Sind die Parameter typisiert, werden diese vom Rayden-Interpreter überprüft. Sind keine Werte für einen Parameter vorhanden, wird die Ausführung mit einem Fehler abgebrochen. 

\SuperPar
Neben einem Typ kann man bei einem Parameter auch noch die Richtung definieren. Die Richtung bezieht sich auf die Gültigkeitsbereiche. In Rayden werden die Richtungen \enword{In}, \enword{Out} und \enword{InOut} unterstützt, wie das Code-Beispiel \ref{prog:parameter} zeigt. \\

\begin{itemize}
\item \textbf{\enword{In}-Parameter}\\
Der \enword{In}-Parameter transferiert einen Wert aus dem Eltern-Gültigkeitsbereich in den Kind-Gültigkeitsbereich. Das ist auch das Standardverhalten, falls keine Richtung bei einem Parameter definiert ist.\\

\item \textbf{\enword{Out}-Parameter}\\
Der \enword{Out}-Parameter ist das genaue Gegenteil zum \enword{In}-Parameter. Dabei wird ein Wert aus dem Kind-Gültigkeitsbereich in den Eltern-Gültigkeitsbereich transferiert. \\

\item \textbf{\enword{InOut}-Parameter}\\
Die dritte Variante ist eine Kombination aus dem \enword{In}-Parameter und dem \enword{Out}-Parameter.\\
\end{itemize}

%%------------------------------------------------------------------------------------------------------
\section{Datentypen von Rayden}

Die Sprache Rayden unterstützt die folgenden Datentypen:

\begin{itemize}
\item \enword{number},
\item \enword{string},
\item \enword{boolean},
\item \enword{variable},
\item \enword{location} und
\item \enword{enumeration}.
\end{itemize}

\SuperPar
Darunter befinden sich einige Standard-Datentypen wie \enword{number}, \enword{string} und \enword{boolean}. 

\begin{program}
\begin{JavaCode}
keyword Open Browser { 
	parameter in browserType as enumeration (IE | FF | CHROME)

	implemented in java -> "selenium.OpenBrowserKeyword"
}
\end{JavaCode}
\caption{Verwendung eines \enword{enumeration}-Parameters}
\label{prog:enum}
\end{program}

\SuperPar
Der Typ \enword{enumeration} wird intern als \enword{string} repräsentiert. Die Laufzeitumgebung sorgt dafür, dass nur die vordefinierten Werte zugewiesen werden dürfen. Diese Überprüfung wird aber nur bei einem Übergang von einem Gültigkeitsbereich in einen anderen Gültigkeitsbereich vorgenommen. Diese Einschränkung ist damit zu erklären, dass ein \enword{enumeration}-Datentyp genau für ein \enword{Keyword} definiert wird. Ein Beispiel dazu sieht man im Code-Ausschnitt \ref{prog:enum}.  

\SuperPar
Ein weiterer spezieller Datentyp ist \enword{location}. Mit diesem Datentyp kann man ein Objekt in einem \enword{Object Repository} referenzieren. Ein Wert dieses Datentyps beginnt immer mit einem @-Symbol. Nachfolgend kann man einen Pfad im \enword{Object Repository} beschreiben, wie man im Beispiel \ref{prog:locator} sieht. Für Abnahmetests ist das Referenzieren von Testobjekten essentiell. Daher bietet Rayden dafür eine Erleichterung. 

\begin{program}
\begin{JavaCode}
Click Left( @PetClinic.PetClinicWeb.Login.Go )
@PetClinic.PetClinicWeb.Login.Go :: Click Left
\end{JavaCode}
\caption{Verwendung des Datentyps \enword{location}}
\label{prog:locator}
\end{program}

\SuperPar
Falls der erste Parameter eines \enword{Keywords} der Datentyp \enword{location} ist, kann man diesen Parameter vor das \enword{Keyword} schreiben. Somit wird das Lesen eines Tests erleichtert. Eine Verwendung dazu findet man ebenfalls im Beispiel \ref{prog:locator}. Diese Spracheigenschaft wird von der Rayden-Laufzeitumgebung wieder in einen klassischen \enword{Keyword}-Aufruf umgebaut.

\begin{program}
\begin{JavaCode}
keyword For Keyword Beispiel {
	For (i, 0, 2) {
		Print("Hello - " + i)
	}
}

keyword For { 
	parameter in var as variable
	parameter in from as number
	parameter in to as number

	implemented in java -> "com.github.thomasfischl.rayden.runtime.keywords.impl.ForKeyword"
}
\end{JavaCode}
\caption{Verwendung des Datentyps \enword{variable}}
\label{prog:variable}
\end{program}

\SuperPar
Der letzte Datentyp ist \enword{variable}. Dieser Datentyp wird verwendet, wenn man den Namen einer Variable an ein \enword{Keyword} übergeben will. Dieser Datentyp beeinflusst die Auswertung von Ausdrücken. Wird ein Ausdruck mit dem Datentyp \enword{variable} typisiert, werden alle Verwendungen von Variablen in diesem Ausdruck nicht ausgewertet. Ein gutes Beispiel dazu ist das \enword{For-Keyword} aus dem Code-Ausschnitt \ref{prog:variable}.

\SuperPar
In diesem Beispiel ist der Parameter \enword{var} als \enword{variable} deklariert. Dadurch wird der Ausdruck \enword{i} nicht ausgewertet, sondern als Zeichenkette der \enword{Keyword}-Implementierung übergeben. Somit kann die Implementierung eine neue Variable mit dem Namen \enword{i} anlegen. Würde man den Parameter \enword{var} mit einem anderen Datentyp versehen, würde die Ausführungseinheit für Ausdrücke versuchen, diese Variable mit einem Wert aus dem Gültigkeitsbereich zu ersetzen. Wird kein Wert für \enword{} gefunden, wird die Ausführung mit einem Fehler abgebrochen.

\SuperPar
Eine Typumwandlung ist in der Sprache Rayden nicht vorgesehen. Es besteht zwar die Möglichkeit, alle Datentypen zu einem \enword{string}-Datentyp umzuwandeln, aber alle anderen Kombinationen sind nicht möglich. In der Implementierung eines \enword{Keywords} können die Werte beliebig konvertiert werden. Die Laufzeitumgebung stellt den Datentyp nur innerhalb der Gültigkeitsbereiche sicher.

%%------------------------------------------------------------------------------------------------------
\section{Verarbeitung von \enword{Keywords} und Ausdrücken}

Im Rayden-System sind Interpreter und \enword{Runtime} für die Ausführung eines Tests zuständig. Dabei wird die Ausführung von \enword{Keywords} und Ausdrücken voneinander getrennt. Die \enword{Keywords} werden von einer \enword{Stack}-Maschine ausgeführt. 

\SuperPar
Die Ausdrücke werden in einer eigenen Ausführungseinheit behandelt. Die Ausführungseinheit verwendet keine \enword{Stack}-Maschine, sondern den rekursiven Abstieg für die Auswertung. Dabei kann diese Einheit entweder untypisiert oder typisiert ausgeführt werden. Diese Eigenschaft zur Typisierung von Parametern wird benötigt, um die Funktionalität einiger Datentypen zu ermöglichen. Darunter fallen die Datentypen \enword{variable} und \enword{enumeration}. Für diese beiden Datentypen muss sich die Ausführungseinheit entweder anders verhalten oder zusätzliche Überprüfungen durchführen. 

%%------------------------------------------------------------------------------------------------------
\section{\enword{Library} und \enword{Bridge}}

Um mit Rayden auch große Testprojekte verwalten zu können, gibt es das Konzept von Bibliotheken (\enword{Libraries}). Eine Bibliothek besteht aus einer Menge von \enword{Keywords}. Es können sowohl \enword{Scripted-}, \enword{Scripted-Compound-} also auch \enword{Compound-Keywords} in einer Bibliothek enthalten sein. Wobei man wahrscheinlich eher \enword{Scripted-} und \enword{Scripted-Compound-Keywords} in einer typischen Bibliothek finden wird.  

\begin{program}
\begin{JavaCode}
keyword For { 
	parameter in var as variable
	parameter in from as number
	parameter in to as number

	implemented in java -> "com.github.thomasfischl.rayden.runtime.keywords.impl.ForKeyword"
}

keyword If { 
	parameter in condition as boolean

	implemented in java -> "com.github.thomasfischl.rayden.runtime.keywords.impl.IfKeyword"
}

keyword Print {
	parameter text
	implemented in java ->"com.github.thomasfischl.rayden.runtime.keywords.impl.PrintKeyword"
}
\end{JavaCode}
\caption{Bibliothek: \enword{stdlibrary.rlg}}
\label{prog:library}
\end{program}

\SuperPar
In einer Bibliothek werden \enword{Keywords} thematisch zusammengefasst. Es wäre beispielsweise möglich, dass es eine Standard-Bibliothek gibt, wie im Code-Beispiel \ref{prog:library} zu sehen ist. In diesem Beispiel sind \enword{For-, If-} und \enword{Print-Keyword}-Definitionen enthalten. Die Datei \enword{stdlibrary.rlg} und das dazugehörige \enword{Java}-Archiv bilden eine Rayden-Bibliothek. 

\SuperPar
Um eine Bibliothek verwenden zu können, muss man diese über die Direktive \enword{import library} einbinden. Ein Beispiel sieht man dazu im Code-Ausschnitt \ref{prog:libraryUsage}. Nachdem die Bibliothek eingebunden wurde, können alle \enword{Keywords} daraus verwendet werden. Für \enword{Keywords} gibt es nur einen Namensraum. Falls es durch das Einbinden von Bibliotheken zu Namenskonflikten kommen sollte, wird die erste Implementierung, die gefunden wird, verwendet. In der Reihenfolge kommen zuerst die aktuellen \enword{Keywords} aus der Datei und danach die Bibliotheken in der Reihenfolge, in welcher diese definiert wurden.

\begin{program}
\begin{JavaCode}
import library "stdlibrary"

keyword Library Beispiel {
	If (1 == 1){
		Print("Condition is true")
	}
	For ("i", 0, 2){
		Print("Hello - " + i)
	}
}
\end{JavaCode}
\caption{Verwendung der \enword{StdLib} Bibliothek}
\label{prog:libraryUsage}
\end{program}

\SuperPar
In Rayden wird zwischen einer \enword{Library} und einer \enword{Bridge} unterschieden. In dieser Ausbaustufe des Rayden-Systems ist die Unterscheidung jedoch nur semantisch. 

\SuperPar
Unter einer \enword{Library} versteht man grundlegende Funktionen wie Schleifen, Verzweigungen und Validierungen. Im Gegensatz dazu besteht eine \enword{Bridge} aus \enword{Keywords}, welche spezifisch für eine Anwendungstechnologie sind. Dazu ist eine \enword{Bridge} auch meistens mit einem Test-Treiber gekoppelt, welcher die Basisfunktionen zur Verfügung stellt. Die \enword{Keywords} kapseln die Funktionalität aus dem Test-Treiber und stellen diese zur Verfügung. Eine \enword{Bridge} kann zum Beispiel das Steuern eines Browsers unterstützen und verwendet dazu Selenium. 


%%------------------------------------------------------------------------------------------------------
\section{\enword{Object Repository}}

Das \enword{Object Repository} stellt eine Abstraktion zur Test-Anwendung her. Alle Testobjekte, welche in einem Test verwendet werden, können in einem \enword{Object Repository} verwaltet werden. In den Tests muss nicht jedes Mal der volle Bezeichner für das Testobjekt verwendet werden, sondern nur eine Referenz darauf. 

\begin{program}
\begin{JavaCode}

objectrepository PetClinic {

	application PetClinicWeb {
		location absolute "/browser"
		
		page Login {
			location "/body/div/div[text='bla']"
			
			button Go {
				location "/btn[text='GO']"
			}
			
			control<Special Button> Cancel {
				location "/div[text='Cancel']"
			}
		
			textfield Username {
				location  "/input[id='username']"
			}
			
			textfield Password {
				location  "/input[id='password']"
			}			
		} 
	}
}
\end{JavaCode}
\caption{\enword{Object Repository}}
\label{prog:or}
\end{program}

\SuperPar
Die Testobjekte werden im \enword{Object Repository} in einem Baum verwaltet und können über diesen auch referenziert werden. Das Beispiel \ref{prog:or} zeigt ein \enword{Object Repository} für eine Webanwendung, welche als Bezeichner einen \enword{XPath}-Ausdruck verwendet.  Der Vorteil davon ist, dass der Bezeichner \enword{location} über die Baumstruktur zusammengebaut wird. Dadurch erspart man sich viel Wartungsaufwand. 

\SuperPar
Typischerweise werden in einem \enword{Object Repository} nur Testobjekte von einer Test-Anwendung zusammengefasst. Werden in einem Test mehrere Anwendungen getestet, sollten dafür unterschiedliche \enword{Object Repositories} angelegt werden.

\begin{program}
\begin{JavaCode}
page Main Page{
	list Owners (index) {
		location  "/ul/ur[$index]"
	}
}

\end{JavaCode}
\caption{Parametrisiertes Testobjekt}
\label{prog:orParam}
\end{program}

\SuperPar
Es ist auch möglich, dass man ein Testobjekt in einem \enword{Object Repository} parametrisiert. Damit können zum Beispiel Listen abgebildet werden, indem der Index als Parameter definiert wird. Ein Beispiel dazu findet man im Code-Ausschnitt \ref{prog:orParam}. Im Bezeichner \enword{location} werden keine Ausdrücke unterstützt. Die Parameter werden über eine Substituierung ersetzt und benötigen daher keinen Datentyp.


%%------------------------------------------------------------------------------------------------------
\section{\enword{Java-Scripting-API}}

Das Rayden-System implementiert die \enword{Java-Scripting-API}. Durch diese Implementierung kann man in jedem \enword{Java}-Programm einen Rayden-Test ausführen. Somit lässt sich das Rayden-System in viele unterschiedliche Szenarien einbinden. 

\begin{program}
\begin{JavaCode}
ScriptEngineManager manager = new ScriptEngineManager();
manager.registerEngineName("RaydenLangScriptEngine", new RaydenScriptEngineFactory());
\end{JavaCode}
\caption{Code-Beispiel: \enword{ScriptEngineFactory} für Rayden registrieren}
\label{prog:registerFactory}
\end{program}

\SuperPar
Um einen Rayden-Test in ein \enword{Java}-Programm einbinden zu können, muss man zuerst die \enword{RaydenScriptEngineFactory} registrieren. Damit gibt man dem \enword{ScriptEngineManager} eine neue Sprache bekannt. Im Code-Beispiel \ref{prog:registerFactory} sieht man eine Möglichkeit, wie man die Rayden-Sprache über einen Namen registrieren kann. Es gibt auch noch andere Möglichkeiten, wie etwa das Registrieren über die Dateiendung.

\begin{program}
\begin{JavaCode}
ScriptEngineManager manager = new ScriptEngineManager();
ScriptEngine engine = manager.getEngineByName("RaydenLangScriptEngine");
Object result =  engine.eval(new FileReader("./test/simple-test.rlg"));
RaydenScriptResult resultObj = (RaydenScriptResult) result;
\end{JavaCode}
\caption{Code-Beispiel: Ausführen eines Rayden-Tests}
\label{prog:runEngine}
\end{program}

\SuperPar
Über die \enword{ScriptEngineFactory} kann der \enword{ScriptEngineManager} eine neue Instanz einer \enword{ScriptEngine} anlegen. Das Code-Beispiel \ref{prog:runEngine} zeigt, wie man einen Test aus einer Datei einliest und diesen über die \enword{ScriptEngine} ausführen lassen kann.

\SuperPar
In diesem Kapitel wurden der Aufbau und die Verwendung des Rayden-Systems beschrieben. Im nächsten Kapitel wird ein Testprojekt mit dem Rayden-System umgesetzt. Dazu werden alle vorher beschriebenen Testmethoden angewendet.


\chapter{Implementierung von Rayden}
\label{cha:Implementierung}

Dieses Kapitel beschreibt die Implementierung von ausgewählten Komponenten des Rayden-Systems. 

\SuperPar
Die Abschnitte \ref{cha:KeywordGrammar} und \ref{cha:StackMachine} zeigen die Grammatik der Sprache Rayden und den Interpretierer für die Ausführung von \enword{Keywords}. Die Abschnitte enthalten Codeausschnitte der Implementierung und Teile der Grammatik.

\SuperPar
Der Abschnitt \ref{cha:Eval} befasst sich mit der Umsetzung und Auswertung von Ausdrücken im Rayden-System. Dazu werden in diesem Abschnitt Auszüge aus der Grammatik der Sprache Rayden und Teile der \enword{RaydenExpressionEvaluator}-Klasse erklärt. Die Klasse \enword{RaydenExpressionEvaluator} ist für die Auswertung der Ausdrücke zuständig. 

\SuperPar
Im Abschnitt \ref{cha:validateKeyword} wird die Validierung von Rayden-Tests gezeigt. Dafür wird das Validierungssystem von xText verwendet. 

\SuperPar
Abgeschlossen wird dieses Kapitel mit dem Abschnitt \ref{cha:implementJSA}, welcher die Integration des Rayden-Systems in die \enword{Java-Scripting-API} zeigt. 

%%------------------------------------------------------------------------------------------------------

\section{Umsetzung der \enword{Keyword}-Grammatik}
\label{cha:KeywordGrammar}

Die Sprache Rayden wurde mit dem xText-Compilerbauwerkzeug umgesetzt. Die Abbildung \ref{fig:keywordGrammar} zeigt einen Auszug aus der Grammatik für die Sprache Rayden. Die Regel \enword{KeywordDecl} beginnt eine Definition eines neuen \enword{Keywords}. Am Beginn der Regel wird die Art des \enword{Keywords} definiert. Eine Beschreibung und Auflistung der Arten ist in Abschnitt \ref{cha:KeywordTypes} enthalten. Danach folgt ein Name für das \enword{Keyword} (\enword{IDEXT}), auf welchen eine geöffnete geschwungene Klammer folgt.

\begin{figure}
\centering
\includegraphics[width=0.9\textwidth]{grammar-keyword-all.png}
\caption{Auszug aus der Grammatik für \enword{Keywords}}
\label{fig:keywordGrammar}
\end{figure}

\SuperPar
Die geschwungenen Klammern definieren den Bereich der \enword{Keyword}-Implementierung. Am Anfang der Implementierung kann eine optionale Beschreibung angeführt werden. Auf diese folgt eine optionale Parameterliste. Ein Parameter wird mit der Regel \enword{ParameterDecl} beschrieben und kann 0 bis \enword{N} Mal wiederholt werden. Eine Parameter-Definition besteht aus dem Schlüsselwort \enword{parameter}, einem Namen, einem optionalen Datentyp und einer Richtung.

\SuperPar
Danach folgt entweder die Bindung an ein Codestück mit der Regel \enword{KeywordScript} oder im Fall eines \enword{Compound Keywords} die \enword{Keyword}-Liste mit der Regel \enword{KeywordList}. Die beiden Angaben sind wiederum optional, um \enword{Keyword}-Rümpfe anlegen zu können. Diese Eigenschaft ist hilfreich, wenn die Testmanagerin oder der Testmanager nur die Struktur festlegen möchte, die Umsetzung des \enword{Keywords} jedoch von anderem Testpersonal vorgenommen wird. 

\SuperPar
Die Regel \enword{KeywordCall} definiert den Aufruf eines \enword{Keywords} in einer \enword{Keyword}-Liste. Die Regel fängt normalerweise mit dem Namen des aufzurufenden \enword{Keywords} an. Danach folgt optional die Parameterliste für den Aufruf eines \enword{Keywords}. Die Regel \enword{KeywordCallParameter} definiert die Parameterliste, welche durch runde Klammern umschlossen ist. Die Parameter können als Liste von \enword{Expr}-Regeln definiert werden und werden durch einen Beistrich separiert. Für die einfachere Verwendung und besserer Lesbarkeit enthält die Regel \enword{KeywordCall} auch noch syntaktischen Zucker. Falls der erste Parameter eines \enword{Keywords} vom Typ \enword{location} ist, kann dieser Parameter vor das \enword{Keyword} geschrieben werden. Somit lässt sich die Implementierung leichter lesen. Der Codeausschnitt \ref{prog:locatorSugar} zeigt dazu die Verwendung dieses syntaktischen Zuckers im Vergleich zur klassischen Verwendung. Am Ende der \enword{KeywordCall}-Regel ist es noch möglich, eine \enword{Keyword}-Liste zu definieren. Diese wird benötigt, falls es sich um ein \enword{Scripted Compound Keyword} oder um ein \enword{Inline Keyword} handelt.

\begin{program}
\begin{JavaCode}
  Type Text (@PetClinic.PetClinicWeb.Login.Username , "max.mustermann")
	@PetClinic.PetClinicWeb.Login.Username :: Type Text ("max.mustermann")
	
	
	Click Left( @PetClinic.PetClinicWeb.Login.Go )
  @PetClinic.PetClinicWeb.Login.Go :: Click Left
\end{JavaCode}
\caption{Syntaktischer Zucker für die Verwendung von \enword{location}-Datentypen}
\label{prog:locatorSugar}
\end{program}


%%------------------------------------------------------------------------------------------------------

\section{Ausführung von \enword{Keywords} mit dem Interpretierer}
\label{cha:StackMachine}

Im vorigen Abschnitt \ref{cha:KeywordGrammar} wurde die Grammatik eines \enword{Keywords} der Sprache Rayden erklärt. Dieser Abschnitt beschäftigt sich mit der Ausführung von \enword{Keywords}. Damit der Interpretierer arbeiten kann, benötigt dieser den Zugriff auf den abstrakten Syntaxbaum. 

\SuperPar
Das Compilerbauwerkzeug xText stellt für den abstrakten Syntaxbaum ein \enword{Eclipse-ECore}-Modell zur Verfügung. Der generierte Compiler ist so konzipiert, dass dieser die gesamte Quelltextdatei einliest und daraus einen abstrakten Syntaxbaum erzeugt. Dieser abstrakte Syntaxbaum steht für die weitere Verarbeitung als \enword{ECore}-Modell zur Verfügung. Einen Auszug aus dem Modell zeigt die Abbildung \ref{fig:AST}. Diese Abbildung zeigt die Modell-Repräsentation der Grammatik-Regeln von Abbildung \ref{fig:keywordGrammar}. Dieser Ausschnitt aus dem Modell stellt die Basis für den Interpretierer dar. 

\begin{figure}[h]
\centering
\includegraphics[width=1\textwidth]{keyword-model-diagramm.png}
\caption{Ausschnitt aus dem Grammatik-Modell}
\label{fig:AST}
\end{figure}

\SuperPar
Der Interpretierer für das Rayden-System ist in der Klasse \enword{RaydenRuntime} implementiert, da der Interpretierer eine Teilkomponente der \enword{Runtime} ist. Der Codeauszug \ref{prog:runtime} zeigt die essentielle Methode \enword{executeKeyword}, welche für die Ausführung von \enword{Keywords} verantwortlich ist. Die Methode wird mit einem \enword{KeywordCall}-Objekt aufgerufen. Dieses Objekt bezeichnet das erste \enword{Keyword}, welches von dem Interpretierer ausgeführt wird. Zuerst werden in der Methode übriggebliebene Elemente vom \enword{Stack} entfernt. Danach werden alle \enword{Reporter-}Objekte über den Start eines neuen Testfalles benachrichtigt. Im nächsten Schritt wird ein neuer Gültigkeitsbereich (\enword{RaydenScriptScope}) angelegt und mit dem \enword{KeywordCall}-Objekt initialisiert. Der Gültigkeitsbereich wird dann auf den leeren \enword{Stack} geladen. 

\SuperPar
Nach der Initialisierung des Interpretierers wird die Abarbeitung gestartet. Es werden nun solange die \enword{Keywords} am \enword{Stack} abgearbeitet, bis der \enword{Stack} leer oder ein Fehler bei der Ausführung eines \enword{Keywords} aufgetreten ist. Der Gültigkeitsbereich repräsentiert einen Aufruf eines \enword{Keywords} und die dazugehörigen Parameter und Variablen. Der Gültigkeitsbereich speichert zusätzlich die aktuelle Position in der \enword{Keyword}-Liste, falls es sich um ein \enword{Compound Keyword} oder \enword{Scripted Compound Keyword} handelt. Über die Methode \enword{getNextKeyword} kann der Interpretierer das nächste \enword{Keyword} aus dem aktuellen Gültigkeitsbereich lesen. Liefert die Methode keinen Wert, ist die Ausführung der \enword{Keywords} für diesen Gültigkeitsbereich zu Ende und wird daher vom \enword{Stack} entfernt.

\begin{program}
\lstinputlisting{samplecode/Runtime.java}
\caption{Codeauszug aus der \enword{RaydenRuntime}-Klasse}
\label{prog:runtime}
\end{program}

\SuperPar
Wurde jedoch ein Wert zurückgeliefert, wird mit der Ausführung fortgefahren. Handelt es sich bei dem Wert um ein \enword{KeywordCall}-Objekt, wird die Methode \enword{executeKeywordCall} aufgerufen. Diese Methode löst den Aufruf des \enword{Keywords} über eine \enword{Lookup}-Tabelle auf. Wurde die passende \enword{Keyword}-Implementierung gefunden, wird ein neuer Gültigkeitsbereich angelegt und auf den \enword{Stack} geladen. Wird in der \enword{Lookup}-Tabelle keine passende Implementierung gefunden, wird die Anwendung in einen Fehlerzustand versetzt und die Ausführung abgebrochen. Handelt es sich jedoch um ein \enword{KeywordDecl}-Objekt wird das \enword{Keyword} ausgeführt.  

\SuperPar
Dabei muss der Interpretierer überprüfen, ob es sich um ein \enword{Scripted Compound Keyword} handelt. Bei einem \enword{Scripted Compound Keyword} muss eine andere Ausführung gewählt werden, da sowohl eine Code-Implementierung, als auch eine \enword{Keyword}-Liste vorhanden sind. Bei allen anderen \enword{Keyword}-Metatypen wird die Methode \enword{executeKeywordDecl} ausgeführt. Diese Methode führt bei einem \enword{Scripted Keyword} das spezifizierte Codestück aus. Bei einem \enword{Compound Keyword} wird die \enword{Keyword}-Liste in den Gültigkeitsbereich geladen.

\SuperPar
Wurden alle Gültigkeitsbereiche am \enword{Stack} erfolgreich abgearbeitet, werden am Ende noch alle \enword{Reporter}-Objekte aufgerufen. Danach wird die Ausführung des Interpretierers beendet.

%%------------------------------------------------------------------------------------------------------
\clearpage

\section{Auswertung von Ausdrücken}
\label{cha:Eval}

Dieser Abschnitt befasst sich mit den Grammatik-Regeln und der Auswertung von Ausdrücken. Die Sprache Rayden unterstützt in einigen Bereichen Ausdrücke. Ein Ausdruck wird in der Grammatik mit der Regel \enword{Expr} beschrieben. Die Abbildung \ref{fig:exprGrammar} zeigt einen Überblick über die Grammatik-Regeln von Ausdrücken. 

\begin{figure}
\centering
\includegraphics[width=0.9\textwidth]{grammar-expr.png}
\caption{Grammatik-Regeln für Ausdrücke}
\label{fig:exprGrammar}
\end{figure}

\SuperPar
Die Regeln für den Ausdruck sind klassisch aufgebaut. Die Operationen sind nach ihrem Vorrang in den Regeln eingearbeitet. Die am stärksten bindenden Operationen befinden sich dadurch in der Nähe der Blätter des Ausdrucksbaums. Die schwach bindenden Operationen befinden sich in der Nähe des Wurzelknotens. Die Blätter repräsentieren die Werte in einem Ausdruck, welche in der Regel \enword{Fact} definiert werden. Die Werte können entweder Konstanten oder Variablen sein und haben einen definierten Datentypen. Die Regel \enword{Fact} hat jedoch keine spezielle Behandlung für \enword{Enumerations}. Der Grund dafür ist, dass \enword{Enumerations} intern als \enword{Strings} verarbeitet werden. Die Validierung der \enword{Enumerations} wird nur beim Initialisieren von Gültigkeitsbereichen durchgeführt.

\SuperPar
Für die Auswertung von Ausdrücken ist im Rayden-System die Klasse \enword{RaydenExpressionEvaluator} zuständig. Der Codeausschnitt \ref{prog:evaluator} gibt einen Überblick über die Klasse \enword{RaydenExpressionEvaluator}. Ein Objekt dieser Klasse wird mit einem Gültigkeitsbereich initialisiert. Der Gültigkeitsbereich wird benötigt, um Variablen bei der Abarbeitung auswerten zu können. 

\SuperPar
Die Auswertung eines Ausdrucks wird mit der Methode \enword{eval(Expr expression, String resultType)} gestartet. Als Parameter für diese Methode werden ein \enword{Expr}-Objekt und eine Zeichenkette übergeben. Das \enword{Expr}-Objekt ist im \enword{ECore}-Modell das Wurzelobjekt für einen Ausdruck. Mit dem zweiten Parameter \enword{resultType} kann man die Auswertung des Ausdrucks typisieren. Als Wert für den Parameter \enword{resultType} werden die Namen der Datentypen verwendet, welche als Konstanten in dieser Klasse vorhanden sind. Wurde ein Typ definiert, wird am Ende der Auswertung noch überprüft, ob das Ergebnis dem geforderten Typ entspricht. Passt der Wert nicht zum Datentyp, wird eine Ausnahme geworfen. Es gibt auch noch einen Spezialfall bei der Typisierung: Wird ein Ausdruck mit dem Typ \enword{variable} parametrisiert, werden keine Variablen im Ausdruck ausgewertet. Diese Eigenschaft wird benötigt, um Variablennamen an ein \enword{Keyword} übergeben zu können.

\SuperPar
Der Codeausschnitt \ref{prog:evaluator} enthält am Ende die Implementierung der \enword{eval}-Methode für ein Objekt vom Typ \enword{Fact}. Diese Methode zeigt, wie die Werte aus dem \enword{ECore}-Modell nach Java konvertiert werden. Die Methode zeigt auch, wie Variablen mithilfe des Gültigkeitsbereichs ausgewertet werden können.

\begin{program}
\lstinputlisting{samplecode/RaydenExpressionEvaluator.java}
\caption{Codeauszug aus dem \enword{RaydenExpressionEvaluator}}
\label{prog:evaluator}
\end{program}

%%------------------------------------------------------------------------------------------------------
\clearpage
\section{Validierung eines Rayden-Tests}
\label{cha:validateKeyword}

Um die Entwicklung und Wartung von Rayden-Tests zu unterstützen, wurden neben einer syntaktischen Validierung von Tests zusätzliche Validierungen hinzugefügt. Für die Umsetzung der Validierungen wurde eine Schnittstelle des xText-Frameworks verwendet. Nachdem eine Datei erfolgreich durch den xText-Compiler geladen wurde, können zusätzliche Validierungen vorgenommen werden. Der Vorteil bei diesem Vorgehen ist, dass in dieser Phase bereits das gesamte \enword{ECore}-Modell zur Verfügung steht. Die Validierungen können somit für Überprüfungen auf das gesamte Modell zugreifen. In dieser Phase ist auch schon sichergestellt, dass es keine syntaktischen Fehler gibt, da diese Fehler bereits vom Syntaxanalysator gefunden wurden.

\begin{program}
\lstinputlisting{samplecode/Validator.java}
\caption{Codeauszug aus dem \enword{RaydenDSLJavaValidator}}
\label{prog:validator}
\end{program}

\SuperPar
Um Validierungen implementieren zu können, wird von xText eine Klasse generiert, welche mit \enword{JavaValidator} endet. Im Fall von Rayden heißt diese Klasse \enword{RaydenDSLJavaValidator}. In dieser Klasse können nun sprachspezifische Validierungen hinzugefügt werden. Jede Methode in dieser Klasse, welche mit einer \enword{@Check}-Annotation gekennzeichnet ist, wird zur Validierung ausgeführt. 

\SuperPar
Das Codebeispiel \ref{prog:validator} zeigt die Validierung für die Verwendung von \enword{Keywords}. Diese Validierung wird für alle \enword{KeywordCall}-Modellelemente aufgerufen. Ein \enword{KeywordCall} stellt einen Aufruf eines \enword{Keywords} dar. Die Validierung überprüft, ob für jeden Aufruf eines \enword{Keywords} auch eine Implementierung vorhanden ist. Das Traversieren des Modells und Suchen aller Modellelemente entfällt, da diese Aufgabe vom xText-Framework bei jedem neuen Aufruf des Compilers durchgeführt wird.

\SuperPar
Im ersten Schritt überprüft die Validierung aus dem Codebeispiel \ref{prog:validator}, ob es sich um ein \enword{Inline Keyword} handelt: \\

\begin{itemize}

\item Falls das \enword{KeywordCall}-Modellelement ein \enword{Inline Keyword} repräsentiert, kann die Validierung beendet werden, da ein \enword{Inline Keyword} direkt in einem \enword{KeywordCall}-Element implementiert ist. \\

\item Falls es sich nicht um ein \enword{Inline Keyword} handelt, wird im nächsten Schritt in den \enword{Lookup}-Tabellen gesucht, ob eine Implementierung für das \enword{Keyword} vorhanden ist. Wird keine Implementierung gefunden, wird eine Warnung ausgegeben. \\

\end{itemize}

\SuperPar
Das Fehlen einer Implementierung liefert nur eine Warnung. Würde diese Validierung einen Fehler liefern, würde der Compiler aufgrund dieses Fehlers abbrechen und es könnten keine Tests ausgeführt werden, obwohl diese nicht von dem Fehler betroffen wären. Diese Warnung soll vielmehr eine Unterstützung für die Testerinnen und Tester sein, um fehlende \enword{Keyword}-Implementierungen zu finden.

%%------------------------------------------------------------------------------------------------------

\section{Integration von Rayden in die \enword{Java-Scripting-API}}
\label{cha:implementJSA}

Das Rayden-System bietet eine Integration in die \enword{Java-Scripting-API}. Die \enword{Java-Scripting-API} ist eine standardisierte Schnittstelle für das Ausführen von Skriptsprachen in Java. Über diese Schnittstelle kann man direkt in einem Java-Programm ein Skript in einer beliebigen Sprache ausführen. Die einzige Einschränkung dabei ist, dass für die Skriptsprache eine \enword{ScriptEngineFactory} registriert werden muss. Die \enword{Java-Scripting-API} ist vergleichbar mit der \enword{Dynamic Language Runtime} \cite{DLR} in \enword{Microsoft .Net}. Mit der \enword{Dynamic Language Runtime} ist es zum Beispiel in \enword{C\#} möglich, ein Python-Skript \cite{Python} auszuführen.

\begin{figure}
\centering
\includegraphics[width=0.7\textwidth]{ScriptEngineFactory.png}
\caption{\enword{ScriptEngineFactory} UML-Klassendiagramm}
\label{fig:scriptEngineFactoryUml}
\end{figure}

\begin{program}
\lstinputlisting{samplecode/ScriptEngine.java}
\caption{Codeauszug aus der \enword{RaydenScriptEngine}}
\label{prog:scriptEngine}
\end{program}

\SuperPar
Für die Integration einer neuen Skriptsprache in die \enword{Java-Scripting-API}, muss man die Schnittstellen \enword{javax.script.ScriptEngineFactory} und \enword{javax.script.ScriptEngine} implementieren. Diese beiden Schnittstellen bilden das Bindeglied zwischen der Java- und der Skriptsprachen-Welt. Die \enword{ScriptEngineFactory}-Schnittstelle liefert Metadaten zu einer Skriptsprache, wie den Namen oder die Versionsnummer. Einen Überblick über die Schnittstelle gibt die Abbildung \ref{fig:scriptEngineFactoryUml}. Die wichtigste Methode der Schnittstelle ist aber \enword{getScriptEngine()}. Diese Methode liefert ein \enword{Script-Engine}-Objekt.

\begin{program}
\lstinputlisting{samplecode/RuntimeLoadFile.java}
\caption{Laden von Rayden-Dateien}
\label{prog:loadFile}
\end{program}

\SuperPar
Über ein \enword{Script-Engine}-Objekt können Skripts ausgeführt werden. Für die Ausführung stehen unterschiedliche Ausprägungen der Methode \enword{eval()} zur Verfügung. Der Skriptcode kann entweder als Zeichenkette oder als \enword{Reader} übergeben werden. Mit einem \enword{Reader} kann man ein Skript direkt aus einer Datei einlesen. Der Codeauszug \ref{prog:scriptEngine} zeigt die Hauptimplementierung der \enword{eval()}-Methode aus der \enword{RaydenScriptEngine}-Klasse. Zu Beginn der Methode wird die Rayden-\enword{Runtime} instanziiert und initialisiert. Die \enword{Runtime} wird benötigt, um einen Rayden-Test ausführen zu können. Im nächsten Schritt wird überprüft, ob ein spezieller \enword{Reporter} verwendet werden soll. Standardmäßig wird die \enword{RaydenXMLReporter}-Implementierung verwendet, welche die gesamten Ausgaben einer Test-Ausführung in einer XML-Datei protokolliert. 

\SuperPar
Damit man in einem Rayden-Test eine Bibliothek verwenden kann, ist es für die Rayden-\enword{Runtime} entscheidend, dass spezifiziert ist, wo die Bibliotheken gefunden werden. Dafür wird ein Arbeitsbereich (\enword{Working Folder}) definiert, in welchem Bibliotheken und andere externe Ressourcen gesucht werden. Der Arbeitsbereich ist standardmäßig das Ausführungsverzeichnis der Java-Anwendung. Der Wert kann aber über einen Kontextparameter verändert werden. Kontextparameter werden verwendet, um Parameter an die \enword{Script Engine} übergeben zu können. 

\SuperPar
Nachdem alle Einstellungen für die Rayden-\enword{Runtime} vorgenommen wurden, wird das Skript mit allen Abhängigkeiten geladen. Dazu wird das Skript mit der Methode \enword{loadRaydenFile()} geladen. Die Implementierung der Methode zeigt der Codeauszug \ref{prog:loadFile}. Wie im Codeauszug dargestellt, gibt es zwei Methoden für das Laden von Rayden-Dateien. Eine Rayden-Datei wird durch die Dateiendung \enword{rlg} identifiziert. Die erste Methode \enword{loadRaydenFile()} ist für das Laden der primären Rayden-Datei zuständig. Mit der zweiten Methode \enword{loadLibraryFile()} werden Bibliotheken geladen. Der Hauptgrund für die zwei unterschiedlichen Methoden ist der, dass zwei \enword{Lookup}-Tabellen für \enword{Keywords} vorhanden sind. Es werden alle \enword{Keywords} aus der primären Rayden-Datei in eine spezielle \enword{Lookup}-Tabelle geladen, sodass nur \enword{Keywords} aus dieser Datei ausgeführt werden können.

\SuperPar
Zum Laden der Rayden-Datei wird der generierte xText-\enword{Parser} verwendet. Im nächsten Schritt wird das Ergebnis des \enword{Parsers} abgefragt und auf Fehler überprüft. Bei einem Fehler wird die Ausführung sofort beendet. Konnte die Datei ohne Fehler geladen werden, liefert der \enword{Parser} eine Instanz des \enword{ECore}-Modells. Das Modell wird durchlaufen und alle \enword{Keyword}-Namen, die gefunden werden, in der \enword{Lookup}-Tabelle gespeichert. Im letzten Schritt werden noch alle Bibliotheken geladen. Das Laden einer Bibliothek funktioniert identisch wie das Laden der primären Rayden-Datei, nur werden in diesem Fall die \enword{Keywords} in eine andere \enword{Lookup}-Tabelle gespeichert. 

\SuperPar
Nachdem das Skript und alle externen Ressourcen erfolgreich geladen wurden, können die Rayden-Tests ausgeführt werden. Dazu stellt die Rayden-\enword{Runtime} die Methode \enword{executeAllTestSuites()} zur Verfügung. Die Methode sucht in der primären \enword{Lookup}-Tabelle nach \enword{Keywords} mit der \enword{Keyword}-Art \enword{testcase}. 

\SuperPar
Die Sprache Rayden unterstützt folgende Arten:

\begin{itemize}
\item Test-Suite (\enword{TestSuite}),
\item Testfall (\enword{TestCase}),
\item Komponententest (\enword{UnitTest}),
\item Integrationstest (\enword{IntegrationTest}),
\item Schnittstellentest (\enword{APITest}),
\item automatisierter Abnahmetest (\enword{AUTest}) und
\item manueller Abnahmetest (\enword{MAUTest}).
\end{itemize}

\SuperPar
Alle gefunden Tests werden im nächsten Schritt sequenziell ausgeführt. Als Resultat der Skriptausführung wird das kumulierte Ergebnis der Tests geliefert. Das Ergebnis kann dann entweder in der Java-Anwendung, welche die Tests gestartet hat, oder nachträglich über die XML-Datei ausgewertet werden.

\SuperPar
In diesem Kapitel wurde gezeigt, wie einige essentielle Komponenten des Rayden-Systems umgesetzt wurden. Im nächsten Kapitel wird erläutert, wie man Tests mit dem Rayden-System schreiben und ausführen kann. Dazu werden unterschiedliche Testmethoden verwendet, um die Stärken von Rayden zu demonstrieren.

\chapter{Umsetzung eines Testprojektes mit Rayden}
\label{cha:Testen}

Es wird gezeigt, wie man ein Testprojekt mithilfe von Rayden umsetzten kann. Dabei wird eine einfache Webanwendung getestet. Zum Beispiel ein Rechner oder kleine Task Anwendung. Dafür werden alle Ebenen von funktionalen Tests durchgeführt.

%%------------------------------------------------------------------------------------------------------

\section{Beispielanwendung}

Für das Evaluieren des Rayden-Systems wird eine Anwendung zum Test benötigt. Bei der Anwendung sollte es sich um eine Webanwendung handeln, um die Unterstützung von Selenium zeigen zu können. Für die Evaluierung hat man sich für die \enword{PetClinic}-Webanwendung entschieden, welche eine Beispielanwendung des Spring-Projekts ist. Die Anwendung mit allen Ressourcen ist öffentlich auf Github unter der Adresse \enword{https://github.com/spring-projects/spring-petclinic/} zugänglich.

\begin{figure}
\centering
\includegraphics[width=0.9\textwidth]{petclinic.png}
\caption{Startseite der Webanwendung PetClinic}
\label{fig:petClinicPage}
\end{figure}

\SuperPar
Bei der \enword{PetClinic}-Anwendung handelt es sich um eine Verwaltungssoftware für eine Tierklinik. Mit der Anwendung können Besuche bei einem Tierarzt protokolliert werden. Dazu gehört die Erfassung der Tierbesitzer mit ihren Haustieren. Zu jedem Haustier werden alle Arztbesuche gespeichert, damit der Krankheitsverlauf dokumentiert ist. Neben den Besitzer und ihren Tieren werden Tierärztinnen und Tierärzte verwaltet. 

\SuperPar
Da der Funktionsumfang der Anwendung überschaubar ist, eignet sich diese ausgezeichnet als Beispielanwendung für die Evaluierung des Rayden-System. In den nächsten Abschnitten werden Test mit unterschiedlichen Testmethoden für die \enword{PetClinic}-Anwendung gezeigt.

\SuperPar
In den folgenden Abschnitte \ref{cha:TestenUnit}, \ref{cha:TestenApi} und \ref{cha:TestenUA} werden Tests für die \enword{Petclinic}-Anwendung vorgestellt. Diese Tests wurden mit drei unterschiedlichen Testmethoden umgesetzt um zu zeigen, wie man die Testmethoden mit dem Rayden-System vereinen kann.

%%------------------------------------------------------------------------------------------------------
\section{Komponententests}
\label{cha:TestenUnit}

Dieser Abschnitt zeigt die Umsetzung eines Komponententests mit dem Rayden-System. Für einen Komponententest wir in Rayden zuerst ein \enword{Keyword} angelegt. Der Codeauszug \ref{prog:unitTest} zeigt die Definition des \enword{Keywords}. Ein Komponententest wird normalerweise als \enword{Scripted-Keyword} umgesetzt und mit dem \enword{Keyword}-Typ \enword{unittest} gekennzeichnet. In diesem Beispiel wird die Komponente \enword{PetTypeFormatter} getestet. Die Komponente ist in der Webanwendung dafür verantwortlich, aus einer Zeichenkette das dazugehörige Domänenobjekt zu liefern und umgekehrt. 

\begin{program}
\begin{JavaCode}
unittest Test PetTypeFormatter {
	''' This unittest verifies the functionality of the 
	    formatter class PetTypeFormatter '''
	implemented in java -> "petclinic.TestPetTypeFormatterKeyword"
}
\end{JavaCode}
\caption{Komponententest \enword{Test PetTypeFormatter}}
\label{prog:unitTest}
\end{program}

\SuperPar
Der Codeausschnitt \ref{prog:unitTestImpl} zeigt die Implementierung des \enword{Scripted-Keywords}. Die Implementierung des Komponententests ist grundlegend gleich mit einem normalen \enword{JUnit}-Test. Die großen Unterschiede sind, dass die Methode nicht mit \enword{@Test} annotiert werden und dass es nur eine Testmethode pro Klasse geben kann. 

\begin{program}
\begin{JavaCode}
public class TestPetTypeFormatterKeyword implements ScriptedKeyword {

  @Override
  public KeywordResult execute(String keyword, KeywordScope scope, RaydenReporter reporter) {
    ClinicService service = new MockClinicService();
    PetTypeFormatter formatter = new PetTypeFormatter(service);

    try {
     Assert.assertEquals("dog", formatter.parse("dog", null).getName());
     Assert.assertEquals("cat", formatter.parse("cat", null).getName());
     Assert.assertEquals("fish",formatter.parse("fish",null).getName());
    } catch (ParseException e) {
      throw new AssertionError(e);
    }
    
    try {
      formatter.parse("hamster", null);
      Assert.fail("No ParseExeption was thrown!");
    } catch (ParseException e) {
    }

    try {
      formatter.parse(null, null);
      Assert.fail("No ParseExeption was thrown!");
    } catch (ParseException e) {
    }

    return new KeywordResult(true);
  }
}
\end{JavaCode}
\caption{Implementierung des \enword{Test PetTypeFormatter} \enword{Keywords}}
\label{prog:unitTestImpl}
\end{program}

\SuperPar
Eine nützliche Erweiterung des Rayden-Systems in der Zukunft wäre eine bessere Integration mit \enword{Unittest-Frameworks} wie \enword{JUnit} oder \enword{TestNG}.

%%------------------------------------------------------------------------------------------------------
\section{Schnittstellentest}
\label{cha:TestenApi}

Bei einem Schnittstellentest werden öffentliche Schnittstellen wie ein \enword{Restful}-Schnittstelle \cite{Rest} getestet. Im Schnittstellentest \ref{prog:integrationTest} wird die \enword{Restful}-Schnittstelle für Tierärzte getestet. Schnittstellentests können entweder als \enword{Compound-Keywords} oder als \enword{Scripted-Keywords} definiert werden. Es hängt ganz davon ab, ob die Implementierung des \enword{Scripted-Keywords} in einem anderen Test wieder verwendet werden kann. 

\begin{program}
\begin{JavaCode}
apitest Test Veterinarians Restful Service {
	'''This keyword checks the restfull service for veterinarian.'''
	
	Verify Json("http://localhost:9966/petclinic/vets.json", 
	            "./demodata/vets.json")
}

keyword Verify Json {
  '''The keyword download the content from the given url. A second 
	   content is loaded from the file. The both contents are parsed
		 into a JSON object tree. If the two trees were equals, the 
		 keyword finish successfully'''
		
	parameter url
	parameter file

	implemented in java -> "petclinic.VerifyJsonKeyword"
}
\end{JavaCode}
\caption{Integrationstest \enword{Test Veterinarians Restful Service}}
\label{prog:integrationTest}
\end{program}

\SuperPar 
Bei diesen Beispiel liefert die Schnittstelle das Ergebnis als einen JSON-Text. Um zu Überprüfen ob die Schnittstelle korrekt funktioniert, wird dieser Text mit einem Text aus einer Demodaten-Datei verglichen. Damit der Test erfolgreich durchläuft, müssen die beiden Texte semantisch Gleich sein. Semantisch Gleich heißt bei einem JSON-Text, dass die enthaltenen Daten gleich sein müssen, aber nicht in welcher Reihenfolge diese serialisiert worden sind.

\begin{program}
\begin{JavaCode}
public class VerifyJsonKeyword implements ScriptedKeyword {
  @Override
  public KeywordResult execute(String keyword, KeywordScope scope, RaydenReporter reporter) {
    String url = scope.getVariableAsString("url");
    String file = scope.getVariableAsString("file");

    try {
      CloseableHttpClient client = HttpClientBuilder.create().build();
      CloseableHttpResponse response = client.execute(new HttpGet(url));
      if (response.getStatusLine().getStatusCode() != 200) {
        return new KeywordResult(false);
      }
      String json = IOUtils.toString(response.getEntity().getContent());

      JsonParser parser = new JsonParser();
      JsonElement o1 = parser.parse(json);
      JsonElement o2 = parser.parse(IOUtils.toString(new FileInputStream(file)));

      return new KeywordResult(o1.equals(o2));
    } catch (IOException e) {
      throw new RuntimeException(e);
    }
  }
}
\end{JavaCode}
\caption{Implementierung des \enword{Verify Json Keywords}}
\label{prog:integrationTestImpl}
\end{program}

\SuperPar
Wenn mehrere Schnittstellen dieser Art getestet werden, ist es sinnvoll, dass man die Funktionalität zum Abfragen und Vergleichen der Daten in ein separates \enword{Keyword} kapselt. Dadurch können andere Tests dieses \enword{Keyword} wiederverwenden. Die Implementierung des \enword{Verify Json Keywords} zeigt das Codebeispiel \ref{prog:integrationTestImpl}. Das Codestück zeigt, dass zuerst über einen \enword{HttpClient} der JSON-Text von einem Server abgefragt wird. Danach wird der JSON-Text mit einem \enword{Parser} in einem Objektbaum transformiert. Dafür wird eine \enword{Parser}-Implementierung aus der \enword{Google-Guava}-Bibliothek verwendet. Der selbe Prozess wird auch mit der Demodaten-Datei durchlaufen. Am Ende gibt es zwei Objektbäume für die JSON-Texte. Für den semantischen Vergleich der beiden Bäume kann die Methode \enword{equals()} der Klasse \enword{JsonElement} verwendet werden. Diese Klasse stammt wiederum aus der \enword{Google-Guava}-Bibliothek.

%%------------------------------------------------------------------------------------------------------
\section{Abnahmetests}
\label{cha:TestenUA}

In diesem Abschnitt wird die letzte Testmethode für die Evaluierung erläutert. Dabei handelt es sich um Abnahmetests, die wohl wichtigste Testmethode für das Rayden-System. Ein Großteil des Rayden-System ist primär für die Unterstützung dieser Testmethode entwickelt worden. Aus diesem Grund enthält dieser Abschnitt auch zwei Umsetzungen von Testmethoden mit Rayden. Als erstes wird der Testfall \enword{Suchen nach einen Tierbesitzer} \ref{cha:TestenUA1} umgesetzt. Bei diesem Testfall wird die Suche der \enword{Petclinic}-Anwendung getestet. Im zweiten Testfall \ref{cha:TestenUA2} wird das Anlegen eines neuen Tierbesitzers gezeigt. 

\SuperPar
Für das Steuern der Browser wurde für beide Abnahmetests die Selenium-Bibliothek verwendet. Aus diesem Grund wird im dritten Teil \ref{cha:TestenSelenium} dieses Abschnittes die Bindung zwischen dem Rayden-System und der Selenium-Bibliothek gezeigt. 

%%------------------------------------------------------------------------------------------------------

\subsection{Abnametest \enword{Suchen nach einen Tierbesitzer}}
\label{cha:TestenUA1}

Bei dem  Abnahmetest im Codeausschnitt \ref{prog:uatest-find} wird die Suchfunktion der \enword{Petclinic}-Anwendung getestet. Dafür wird der Testfall als erstes in zwei \enword{Compound-Keywords} \enword{Find a specific Pet Owner} und \enword{Check Owner Details} aufgeteilt. Neben den beiden \enword{Keywords} werden noch zwei weiter \enword{Keywords} \enword{Prepare Browser} und \enword{Cleanup Browser}  benötigt, welche für das Starten und Stoppen des Browsers zuständig sind. 

\begin{program}
\lstinputlisting{samplecode/uatest-findpetowner.rlg}
\caption{Abnametests: \enword{Suchen nach einen Tierbesitzer}}
\label{prog:uatest-find}
\end{program}


\begin{program}
\lstinputlisting{samplecode/or-findpetowner.rlg}
\caption{Codeauszug aus dem \enword{Object-Repository} für den Testfall \enword{Suchen nach einen Tierbesitzer}}
\label{prog:or-find}
\end{program}

\SuperPar
Das \enword{Keyword} \enword{Find a specific Pet Owner} führt die Suche nach einem Tierbesitzer mit dem Name \enword{Davis} aus. Dafür navigiert das \enword{Keyword} auf die Seite für Tierbesitzer und startet die Suche nach allen Besitzern. Danach werden auf der Ergebnisseite der Suche alle Besitzer aufgelistet, welche in der Anwendung vorhanden sind. Danach wird mithilfe der integrierten Suche nach dem Namen \enword{Davis} gesucht und die Detailseite des Tierbesitzers geöffnet. Wurde die Detailseite erfolgreich geöffnet ist das \enword{Keyword} am Ende. Die Validierung der Daten werden von dem nächsten \enword{Keyword} \enword{Check Owner Details} durchgeführt.

\SuperPar
Dem \enword{Keyword} \enword{Check Owner Details} werden die zu überprüfenden Daten als Parameter übergeben. Das \enword{Keyword} überprüft jedes Datum mit dem \enword{Scripted-Keyword} \enword{Verify Text}. Dem \enword{Scripted-Keyword} werden zwei Parameter übergeben. Der erste Parameter ist ein \enword{Locator}, welcher ein Element auf der Webseite definiert. Von diesem Element wird der Text abgefragt und mit dem zweiten Parameter verglichen. Der zweite Parameter ist eine Zeichenkette mit dem erwarteten Wert. Sind die Wert nicht gleich, wird der Test mit einem Fehler abgebrochen. 

\SuperPar
Damit in den \enword{Keywords} keine \enword{XPath}-Ausdrücke vorkommen müssen, wird ein \enword{Objekt-Repository} verwendet. Einen Auszug mit den wichtigsten Objekten für diesen Abnahmetest zeigt das Codebeispiel \ref{prog:or-find}. Das Codebeispiel enthält die Objekte für die Suchergebnisseite und die Detailseite für Tierbesitzerinnen und Tierbesitzer.

\todo OR - Bild mit Ergebnis Webseite hinzufügen

%%------------------------------------------------------------------------------------------------------

\subsection{Abnametest \enword{Anlegen eines neuen Tierbesitzer}}
\label{cha:TestenUA2}


Der zweite Abnahmetest testet den Anwendungsfall \enword{Anlegen eines neuen Tierbesitzer}. Diese Abnahmetest wird direkt im Abnahmetest-\enword{Keyword} aus spezifiziert. Als erstes wird wiederum die Testumgebung mit dem \enword{Keyword} {Preapre Browser} vorbereitet. Im nächsten Schritt navigiert der Test auf die Tierbesitzerseite und klick auf die \enword{Add Owner} Schlatfläche. 


\SuperPar
Danach werden alle benötigten Daten für eine neue Tierbesitzerin oder Tierbesitzer im Formular eingegeben. Dazu wird das \enword{Keyword} \enword{Type Text} verwendet. Diese \enword{Scripted-Keyword} schreibt mithilfe der \enword{Selenium}-Bibliothek einen Text in ein Textfeld auf der Webseite. Der Codeauszug \ref{prog:or-create} zeigt die dazugehörige Objekt im \enword{Object-Repository}.  Der Auszug zeigt die Seite \enword{Edit Owner Page}, welche das Formular für das Anlegen und Änderen einer Besitzerin oder eines Besitzers zeigt.

\SuperPar
Das \enword{Add new pet owner} \enword{Keyword} ist ein gutes Beispiel, um zu zeigen, wie man mithilfe des \enword{Object-Repository} einen leserlichen Test schreiben kann. Der Test setzt auschließlich auf \enword{Locators}  und verzeicht auf die direkte Verwendung von \enword{XPath}-Ausdrücken. 

\begin{program}
\lstinputlisting{samplecode/or-createpetowner.rlg}
\caption{Codeauszug aus dem \enword{Object-Repository} für den Testfall \enword{Anlegen eines neuen Tierbesitzer}}
\label{prog:or-create}
\end{program}

\clearpage
%%------------------------------------------------------------------------------------------------------

\subsection{\enword{Keywords} aus der \enword{Selenium}-Bibliothek}
\label{cha:TestenSelenium}

In diesem Abschnitt werden ausgewählte \enword{Selenium Keywords} aus den Abhnametests beschrieben. Diese \enword{Keywords} wurden alle in der Datei \enword{selenium.rlg} angelegt. Die Datei \enword{selenium.rlg} stellt somit die \enword{Bridge} zwischen Rayden und \enword{Selenium} dar. Das Codestück \ref{prog:selenium} zeigt drei \enword{Scripted-Keywords}, welche einen guten Überblick in die Integration von \enword{Selenium} in das Rayden-System zeigen. 

\begin{program}
\lstinputlisting{samplecode/selenium.rlg}
\caption{Codeauszug aus der Selenium \enword{Keyword}-Bibliothek}
\label{prog:selenium}
\end{program}

\SuperPar
Als erstes wird das \enword{Open Browser Keyword} \ref{prog:openBrowserKeyword} beschrieben. Dieses \enword{Scripted-Keyword} ist essentiel für die Verwendung von Selenium, da dieses \enword{Keyword} die Testumgebung vorbereit. Als Parameter werden der Browsertyp und eine \enword{URL} übergeben. Der Browsertyp definiert den Browser, welcher gestartet werden soll. Das kann zum Beispiel der \enword{Internet Explorer}, \enword{Firefox} oder \enword{Google Chrome} sein. Mit dem Browsertyp Parameter wird eine \enword{Webdriver}-Objekt angelegt. Dieses Objekt dient als Schnittstelle zwischen dem Test und Browser-Instanz. Jedes neue \enword{Webdriver}-Objekt startet einen Browser. Der zweite Parameter \enword{URL} definiert die Startseite. Über die Methode \enword{navigate()} der Klasse \enword{Webdriver} kann man auf eine neue Seite im Browser navigieren. 

\begin{program}
\begin{JavaCode}
public class OpenBrowserKeyword implements ScriptedKeyword {

	@Override
	public KeywordResult execute(String keyword, KeywordScope scope, RaydenReporter reporter) {
		String browserType = scope.getVariableAsString("browserType");
		String url = scope.getVariableAsString("url");
		WebDriver driver = Selenium.getInstance().initializeDriver(browserType);
		driver.navigate().to(url);
		return new KeywordResult(true);
	}
}
\end{JavaCode}
\caption{Implementierung des \enword{Open Browser Keywords}}
\label{prog:openBrowserKeyword}
\end{program}

\SuperPar
Bei dem nächsten \enword{Keyword} handelt es sich um das \enword{Click Keyword}, welches im Codestück \ref{prog:clickKeyword} gezeicht wird. Mit diesem \enword{Keyword} kann man einen Klick auf der Webseite auslösen. Um die richtige Position für den Klick herausfinden zu können, wird dem \enword{Keyword} ein \enword{Locator} übergeben. Dieser \enword{Locator} beschreibt ein Element auf der Webseite. Die Auswertung dieses \enword{Locators} wird von der Methode \enword{findElement()} übernommen. Die Methode greift auf das aktuelle \enword{WebDriver}-Objekt zu um das Element zu finden. Diese Aktion könnte man auch direkt über das \enword{WebDriver}-Objekt ausführen, jedoch enthält die Methode \enword{findElement()} aus der \enword{Selenium} Klasse einen zusätzlichen Wiederholungsfunktion im Fehlerfall. Die Funktionaltiät ist für einen stabilen Abnahmetest wichtig, da dieses Synchronisierungsprobleme mit der Webanwendung reduziert. 

\SuperPar
Man spricht von einem Synchronisierungsfehler bei einer Webseite, wenn eine Aktion ausgeführt wird, obwohl die Anwendung noch nicht fertig geladen ist. Dieser Fehler tritt sehr häufig bei stark dynamischen Webseiten auf. Die einfachste und auch primitivste Lösung für das Problem ist die mehrmalige Auswertung des \enword{XPath}-Asudruckes, falls dieser kein Element findet. Genau dieser Ansatz ist in der \enword{findElement()} Methode der \enword{Selenium} Klasse implementiert.

\begin{program}
\begin{JavaCode}
public class ClickKeyword implements ScriptedKeyword {

  @Override
  public KeywordResult execute(String keyword, KeywordScope scope, RaydenReporter reporter) {
    RaydenExpressionLocator locator = (RaydenExpressionLocator) scope.getVariable("locator");
    reporter.log("Click on '" + locator + "'");
    Selenium.getInstance().findElement(locator.getEvalLocator()).click();
    return new KeywordResult(true);
  }
}
\end{JavaCode}
\caption{Implementierung des \enword{Click Keywords}}
\label{prog:clickKeyword}
\end{program}



\begin{program}
\begin{JavaCode}
public class VerifyTextKeyword implements ScriptedKeyword {

  @Override
  public KeywordResult execute(String keyword, KeywordScope scope, RaydenReporter reporter) {
    RaydenExpressionLocator locator = (RaydenExpressionLocator) scope.getVariable("locator");
    String text = scope.getVariableAsString("text");
    WebElement element = Selenium.getInstance().findElement(locator.getEvalLocator());
    String elementText = element.getText();
    reporter.log("Verify Text: '" + text + "'='" + elementText + "'");
    return new KeywordResult(text.equals(elementText));
  }
}
\end{JavaCode}
\caption{Implementierung des \enword{Verify Text Keywords}}
\label{prog:verifyTextKeyword}
\end{program}

\todo

\clearpage

\section{Testdokumentation}

TODO !!!

\chapter{Zusammenfassung}
\label{cha:Zusammenfassung}


\todo

Diskussion, Erfahrungen, weitere Arbeiten
TODO !!!


%%%----------------------------------------------------------
%%%Anhang
\appendix
%%\include{anhang_a}	% Technische Ergänzungen
%%\include{anhang_b}	% Inhalt der CD-ROM/DVD
%%\include{anhang_c}	% Chronologische Liste der Änderungen
%%\include{anhang_d}	% Quelltext dieses Dokuments

%%%----------------------------------------------------------
\MakeBibliography
%%%----------------------------------------------------------

%%%Messbox zur Druckkontrolle
%%\include{messbox}

\end{document}
