\chapter{Einleitung}
\label{cha:Einleitung}

\section{Zur Projektbezeichnung Rayden}

Der Begriff \enword{Rayden} ist abgeleitet von dem japanischen Wort \enword{Raijin}, welches im japanischen Volksglauben der Name des Donner-Gotts ist. In der westlichen Welt wird der Name aber meist \enword{Raiden} geschrieben, woraus für diese Arbeit die Bezeichnung \enword{Rayden} abgeleitet wurde. Die Abbildung \ref{fig:logo} zeigt das Logo für Rayden, welches von Agnes Fischl gestaltet wurde.

\begin{figure}[h]
\centering
\includegraphics[width=0.3\textwidth]{logo.png}
\caption{Rayden-Logo}
\label{fig:logo}
\end{figure}


%%------------------------------------------------------------------------------------------------------

\section{Motivation}

Viele Softwareunternehmen haben in den letzten Jahren große Testabteilungen aufgebaut. Der Fokus in diesen Abteilungen liegt sehr häufig noch auf dem manuellen Testen der Benutzeroberflächen. Dabei müssen für jede neue Version der Software viele manuelle Schritte durchlaufen werden. Dieser Vorgang ist sehr zeit- und kostenintensiv. Durch den Vormarsch neuer Entwicklungsmethoden und wegen des starken Kostendrucks stehen viele dieser Abteilungen vor einem Problem. Auf der einen Seite müssen sie Kosten einsparen, auf der anderen Seite werden die Release-Zyklen immer kürzer, was einen noch größeren Aufwand bedeutet. In diesem Spannungsfeld überlegen viele dieser Unternehmen, ihre manuellen Tests zu automatisieren, um dadurch langfristig Zeit und Geld zu sparen. 

\SuperPar
Dieser Transformationsprozess stellt die Organisationen vor große Herausforderungen. Sie haben tausende Stunden von Expertenwissen in die manuellen Tests investiert. Für die Automatisierung steht jedoch selten derselbe Umfang an Zeit und Geld zur Verfügung. Auch muss der Prozess meistens parallel zu den bestehenden manuellen Tests vollzogen werden, da man kaum eine vollständige Umstellung auf einmal erledigen kann.

\SuperPar
Um diesen Prozess für die Testabteilung zu erleichtern, benötigt es ein mehrschichtiges Test-\enword{Framework}, das in der Lage sein muss, bestehende manuelle Tests wiederverwenden zu können. Dabei darf die Lesbarkeit der manuellen Tests aber nicht verloren gehen, da diese in Ausnahmefällen noch von einer Testerin oder einem Tester manuell durchgeführt werden müssen.

%%------------------------------------------------------------------------------------------------------

\section{Problemstellung}

Viele Testabteilungen arbeiten heutzutage noch größtenteils mit manuellen Tests. Diese Tests sind über Jahrzehnte gewachsen und es wurden tausende von Stunden in die Erstellung und Wartung investiert. Diese Testabteilungen bestehen aus vielen Testerinnen und Testern, welche die Tests für jede neue Version einer Software manuell ausführen. Es kommt nicht selten vor, dass aus Zeitgründen nicht alle Tests für jede Version ausgeführt werden können. Diese Situation hat sich durch den Einsatz von agilen Entwicklungsprozessen und kürzeren Release-Zyklen noch deutlich verschärft. 

\SuperPar
Diese Entwicklungen machen es notwendig, dass sich Testabteilungen immer öfter mit dem Thema der Testautomatisierung auseinandersetzen müssen. 

\SuperPar
Das führt zu folgenden Herausforderungen für die Testabteilungen:\\

\begin{enumerate}

\item Für die Automatisierung der Tests steht oft nur ein geringes Budget zur Verfügung.\\
	
\item Das Wissen aus den manuellen Test darf nicht verloren gehen.\\
	
\item Während des Migrationsprozesses und auch danach muss es möglich sein, dass man automatisierte Tests manuell ausführen kann. Diese Eigenschaft kann notwendig werden, um fehlgeschlagene automatisierte Testausführungen nachträglich manuell verifizieren zu können.\\

\item Die bestehenden Automatisierungslösungen sind oft sehr technisch aufgebaut. Jedoch findet man in typischen Testabteilungen nur wenige Entwicklerinnen und Entwickler, welche mit diesen Lösungen arbeiten können.

\end{enumerate}

\SuperPar
Um alle Herausforderungen dieser Liste zu adressieren, reicht \enword{eine} technische Lösung heutzutage nicht mehr aus. Ein Test-\enword{Framework} in diesem Umfeld muss auf vielen unterschiedlichen Ebenen ansetzen und Unterstützung bieten. 

\SuperPar
Eine mögliche Lösung für Testabteilungen wäre eine Testmethode, welche es auch Testerinnen und Testern ohne fundierte Programmierkenntnisse ermöglicht, automatisierte Tests zu erstellen. Zusätzlich muss es mit dieser Testmethode möglich sein, bestehende manuelle Tests wieder zu verwenden. Schlussendlich müssen die automatisierten Tests noch immer in einem Format vorliegen, das es ermöglicht, diese auch manuell von einer Testerin oder einem Tester auszuführen.

%%------------------------------------------------------------------------------------------------------

\section{Zielsetzung}

Das Ziel dieser Arbeit ist es, die Fähigkeiten eines \enword{Keyword-Driven-Testing}-Ansatzes mit einem \enword{Object Repository}  zu kombinieren. Im Zuge der Implementierung soll ein neues Test-\enword{Framework} entwickelt werden, welches den Ansatz von \enword{Keyword-Driven Testing} verwendet. Für das \enword{Framework} soll eine neue Sprache entwickelt werden, welche die Bedürfnisse nach einer einfachen und gut lesbaren Sprache befriedigt. Zusätzlich soll die Sprache eine gute Unterstützung für das \enword{Object Repository} liefern.

\SuperPar
Im nächsten Kapitel werden die Grundlagen und verwendeten Technologien detailliert beschrieben.