\chapter{Konzept}
\label{cha:Konzept}

\section{Ablauf eines Testprojekts}

TODO !!!

\subsection{Rollen in einem Testprojekt}

Testmanager, Entwickler, Tester

TODO !!!

\subsection{Der Testfall}

TODO !!!

\subsection{Komponenten- und Integrationstests für Testfälle}

TODO !!!

\subsection{Manuelle Abnahmetests für Testfälle}

TODO !!!

\subsection{Automatisieren von manuellen Abnahmetests}

TODO !!!

\subsection{Testdokumentation}

TODO !!!

\section{Die Entwicklung der Testautomatisierung}

\subsection{Record-Replay Testing}

\subsection{Functional Testing}

\subsection{Data-Driven-Testing}

\subsection{Keyword-Driven-Testing}

Erklären was KDT ist! 
Tests sind Daten.
Daten werden von einer Engine ausgewertet.

TODO !!!

\section{Robo-Framework: Ein Keyword-Driven-Testing-Framework}

TODO !!!

\section{Rayden}

Idee von Rayden. Weiterentwicklung von KDT.
Erklärung der Nachteile von KDT und wie man das mit Rayden lösen möchte.
Baustein Metapher.
TODO !!!

