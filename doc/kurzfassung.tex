\chapter{Kurzfassung}

In den letzten Jahren ist das Automatisieren von Tests wieder in den Fokus von Testmanagerinnen und Testmanagern gerückt. Gründe dafür sind die Verkürzung der Release-Zyklen und ein immer größerer Kostendruck. Daher stehen viele Testabteilungen vor dem Problem, ihre manuellen Tests zu automatisieren.

\SuperPar
Ein Lösungsansatz dafür ist der \enword{Keyword-Driven-Testing}-Ansatz, welcher sich in letzter Zeit großer Beliebtheit erfreut. Für diesen Testansatz wurden einige Open-Source-Werkzeuge, aber auch kommerzielle Lösungen entwickelt. Jedoch hat dieser Ansatz neben vielen Vorteilen auch einige Nachteile: Je größer die Projekte werden, desto schwieriger wird die Verwaltung der Tests, da es nur wenige Werkzeuge gibt, welche mit großen Testprojekten umgehen können. 

\SuperPar
Dieser Ausgangspunkt stellt die Motivation für diese Masterarbeit dar, in der ein neues System mit dem Namen Rayden entwickelt wird, welches den \enword{Keyword-Driven-Testing}-Ansatz umsetzt. Jedoch setzt diese Lösung auf einen Compiler, um eine bessere Unterstützung für die Verwenderinnen und Verwender bieten zu können. In Rayden wird auch das Konzept eines \enword{Object Repositories} integriert, das dabei helfen soll, Abnahmetests leichter und besser lesbar zu schreiben.

\SuperPar
Um die Fähigkeiten von Rayden zu zeigen, wird eine Web-Anwendung mit diesem System getestet. Dabei wird gezeigt, wie man unterschiedliche Testmethoden mit Rayden vereinen kann und welche besonderen Stärken im Bezug auf Abnahmetests existieren.