\chapter{Kurzfassung}

In den letzten Jahren ist das Automatisieren von Tests wieder in den Fokus von Testmanagerinnen und Testmanagern gerückt. Gründe dafür sind die Verkürzung der Release-Zyklen und ein immer größerer Kostendruck. Daher stehen viele Testabteilung vor dem Problem ihre manuellen Tests zu automatisieren.

\SuperPar
Ein Lösungsansatz dafür ist der \enword{Keyword-Driven-Testing}-Ansatz, welcher sich in letzter Zeit großer Beliebtheit erfreut. Für diesen Testansatz wurden einige Open-Source-Anwendungen aber auch kommerzielle Lösungen entwickelt. Jedoch hat dieser Ansatz neben vielen Vorteile auch einige Nachteile. Je größer die Projekt werden, desto schwieriger wird die Verwaltung da es nur wenig Werkzeuge für diese Anwendungen gibt. 

\SuperPar
Dieser Ausgangspunkt stellt die Motivation für diese Masterarbeit. In dieser Mastarbeit wird ein neues System entwickelt, welches den \enword{Keyword-Driven-Testing}-Ansatz umsetzt. Jedoch setzt diese Lösung auf einen Compiler um eine bessere Unterstützung für die Verwenderinnen und Verwender liefern zu können. In das System wird auch das Konzept eines \enword{Object Repositories} integriert. Das Konzept soll dabei helfen, Abnahmetests leichter und besser lesbar zu schreiben.

\SuperPar
Um die Fähigkeiten dieses neu entwickelten Systems zu zeigen, wird einen Webanwendung mit diesem System getestet. Dabei wird gezeigt, wie man unterschiedliche Testmethoden mit diesem System vereinen kann und welche besonderen Stärken im Bezug auf Abnahmetests existieren.