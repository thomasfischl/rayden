\chapter{Problemanalyse}
\label{cha:Problemanalyse}

Die Testabteilung wird durch die Umstellung der Softwareentwicklungsprozesse auf agile Entwicklungsmethoden auf eine große Probe gestellt. Durch die Verwendung einer agilen Entwicklungsmethode werden die Release-Zyklen von Anwendungen deutlich reduziert. Das hat zur Folge, dass die Testabteilung einen deutlich höheren Testaufwand bewerkstelligen muss. Dem gegenüber steht die Testabteilung unter einem immer größer werdenden Kostendruck. 

\SuperPar
Aus diesem Grund entscheiden sich viele Testmanagerinnen und Testmanager dafür, ihre manuellen Testabläufe zu automatisieren. Jedoch können die Testabteilungen bei dem Aufbau von automatisierten Tests nicht von Grund auf neu beginnen. In den manuellen Tests stecken jahrelange Entwicklungszeit und Wissen, welches für die automatisierten Tests wieder verwendet werden muss, um bei der Automatisierung erfolgreich zu sein. 

\SuperPar
Ein anderes Problem von der Testabteilung ist, dass diese über keine bis wenige Entwicklerinnen und Entwickler verfügt. Eine klassische Testabteilung besteht normalerweise aus Personen, welche keine fundierten Programmierkenntnisse besitzen. Aus diesem Grund ist die Testabteilung entweder auf die Mithilfe der Entwicklungsabteilung angewiesen oder muss den Anteil an Entwicklerinnen und Entwicklern aufstocken, was wiederum eine Kostensteigerung verursacht.

\SuperPar
Eine mögliche Option für eine Testabteilung wäre eine Testmethode, welche es auch Testerinnen und Testern ohne fundierte Programmierkenntnisse ermöglicht, automatisierte Tests zu erstellen. Zusätzlich muss es mit der Testmethode möglich sein, bestehende manuelle Tests wieder zu verwenden. Schlussendlich müssen die automatisierten Tests noch immer in einem Format vorliegen, das ermöglicht, diese auch manuell von einer Testerin oder einem Tester auszuführen.